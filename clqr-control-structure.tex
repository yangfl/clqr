% Copyright (C) 2008, 2009, 2010, 2011, 2012, 2014 Bert Burgemeister
%
% Permission is granted to copy, distribute and/or modify this
% document under the terms of the GNU Free Documentation License,
% Version 1.2; with no Invariant Sections, no Front-Cover Texts and
% no Back-Cover Texts. For details see file COPYING.
%

%%%%%%%%%%%%%%%%%%%%%%%%%%%%%%%%%%%%%%%%%%%%%%%%%%
\section{控制结构}
%%%%%%%%%%%%%%%%%%%%%%%%%%%%%%%%%%%%%%%%%%%%%%%%%%

%%%%%%%%%%%%%%%%%%%%%%%%%%%%%%%%%%%%%%%%%%%%%%%%%%
\subsection{谓词}
%%%%%%%%%%%%%%%%%%%%%%%%%%%%%%%%%%%%%%%%%%%%%%%%%%
\begin{LIST}{1cm}

  \IT{(\FU*{EQ} \VAR{foo bar})}
  {
  当 \VAR{foo} 和 \VAR{bar} 相同(identical)时为 \retval{\T}。
  }

  \IT{(\FU*{EQL} \VAR{foo bar})}
  {
  当 \VAR{foo} 和 \VAR{bar}
  相同,或相等 \kwd{character},或同类同值的
  \kwd{number}s 时为 \retval{\T}。
  }

  \IT{(\FU*{EQUAL} \VAR{foo bar})}
  {
  当 \VAR{foo} 和 \VAR{bar}
  \FU{EQL},或是等价的 \kwd{pathname}s,或是 cars 和 cdrs
  \FU{equal} 的 \kwd{cons}es,或是在填充指针之内的元素 \FU{eql} 的
  \kwd{string}s 或 \kwd{bit-vector}s 时为 \retval{\T}。
  }

  \IT{(\FU*{EQUALP} \VAR{foo bar})}
  {
  当 \VAR{foo} 和 \VAR{bar} 相同,或是忽略大小写时相同的
  \kwd{character},或是忽略类型时值相等的
  \kwd{number}s,或是等价的 \kwd{pathname}s,或是形状相同、元素 \FU{equalp}
  的 \kwd{cons}es 或 \kwd{array}s,或是元素
  \FU{equalp}、类型一致的结构,或是大小相等、\kwd{:test}
  函数相同、键对 \kwd{:test}
  函数相等、元素 \FU{equalp} 的
  \kwd{hash-table}s 时为 \retval{\T}。
  }

  \IT{(\FU*{NOT} \VAR{foo})\qquad\qquad\qquad}
  {
  当 \VAR{foo} 为 \NIL\ 时为 \retval{\T},否则为 \retval{\NIL}。
  }

  \IT{(\FU*{BOUNDP} \VAR{symbol})\qquad\qquad}
  {
    当 \VAR{symbol} 为特殊变量时为 \retval{\T}。
  }

  \IT{(\FU*{CONSTANTP} \VAR{foo} \Op{\VAR{environment}\DF{\NIL}})}
  {
  当 \VAR{foo} 为常量时为 \retval{\T}。
  }

  \IT{(\FU*{FUNCTIONP} \VAR{foo})\qquad\qquad}
  {
  当 \VAR{foo} 为 \kwd{function} 时为 \retval{\T}。
  }

  \IT{(\FU*{FBOUNDP} \xorGOO{\VAR{foo}\\
      (\kwd{setf } \VAR{foo})}{\}})}
  {
  当 \VAR{foo} 为全局函数或宏时为 \retval{\T}。
  }


\end{LIST}


%%%%%%%%%%%%%%%%%%%%%%%%%%%%%%%%%%%%%%%%%%%%%%%%%%
\subsection{变量}
%%%%%%%%%%%%%%%%%%%%%%%%%%%%%%%%%%%%%%%%%%%%%%%%%%

\begin{LIST}{1cm}

  \IT{(\xorGOO{%
      \MC*{DEFCONSTANT}\\
      \MC*{DEFPARAMETER}}{\}} \NEV{\VAR{foo}} \VAR{form}
      \Op{\NEV{\VAR{doc}}})}
  {
  将 \VAR{form} 的值赋给全局常量/动态变量 \retval{\VAR{foo}}。
  }

  \IT{(\MC*{DEFVAR} \NEV{\VAR{foo}} \OP{\VAR{form}
      \Op{\NEV{\VAR{doc}}}})}
  {
  除非已绑定,将 \VAR{form} 的值赋给动态变量
  \retval{\VAR{foo}}。
  }

  \IT{(\xorGOO{\MC*{SETF}\\
      \MC*{PSETF}}{\}} \Goos{\VAR{place}
      \VAR{form}})}
  {
  顺序/平行将 \VAR{place}s 设为 \VAR{form}s 的主值。返回 \retval{最后一个\
    \VAR{form} 的所有值}/\retval{\NIL}。
  }

  \IT{(\xorGOO{\SO*{SETQ}\\
      \MC*{PSETQ}}{\}} \Goos{\VAR{symbol}
      \VAR{form}})}
  {
  顺序/平行将 \VAR{symbol}s 设为 \VAR{form}s 的主值。返回 \retval{最后一个\
    \VAR{form} 的值}/\retval{\NIL}。
  }

  \IT{(\FU*{SET} \DES{\VAR{symbol}} \VAR{foo})}
  {
  将 \VAR{symbol} 的值单元设为 \retval{\VAR{foo}}。已废弃。
  }

  \IT{(\MC*{MULTIPLE-VALUE-SETQ} \VAR{vars} \VAR{form})}
  {
  将 \VAR{vars} 的各元素分别设为
  \VAR{form} 的所有值。返回 \retval{\VAR{form} 的主值}。
  }

  \IT{(\MC*{SHIFTF} \RP{\DES{\VAR{place}}} \VAR{foo})}
  {
  将 \VAR{place}s 的值左移,将 \VAR{foo} 设在
  \VAR{place} 的最右位置,返回 \retval{第一个\ \VAR{place}}。
  }

  \IT{(\MC*{ROTATEF} \OPn{\DES{\VAR{place}}})}
  {
  向左旋转 \VAR{place}s 的值,原来的第一个值变成
  \VAR{place} 的最后一个值。返回 \retval{\NIL}。
  }

  \IT{(\FU*{MAKUNBOUND} \DES{\VAR{foo}})}
  {
  若存在,删除特殊变量 \retval{\VAR{foo}}。
  }

  \label{:property_lists}
  \IT{\arrGOO{(\FU*{GET} \VAR{ symbol} \VAR{ key }
      \OP{\VAR{default}\DF{\NIL}})\\ 
      (\FU*{GETF} \VAR{ place} \VAR{ key }
      \OP{\VAR{default}\DF{\NIL}})}{.}}
  {
  \VAR{symbol}/\VAR{place} 上的属性表中存储的 \retval{第一个\ \VAR{key} 项},若不存在
  \VAR{key},则为 \retval{\VAR{default}}。可
  \kwd{setf}。
  }

  \IT{(\FU*{GET-PROPERTIES} \VAR{property-list} \VAR{keys})}
  {
  选择 \VAR{property-list} 中第一个匹配 \VAR{keys}
  中键的项目,并返回其 \retval{键}、其 \retvalii{值}
  及 \retvaliii{剩余的\
    \VAR{property-list}}。若 \VAR{property-list} 中无匹配,返回
  \retval{\NIL}、\retvalii{\NIL}、\retvaliii{\NIL}。
  }

  \IT{\arrGOO{(\FU*{REMPROP } \DES{\VAR{symbol}} \VAR{ key})\\
      (\MC*{REMF } \DES{\VAR{place}} \VAR{ key})}{.}}
  {
    从 \VAR{symbol}/\VAR{place} 上的属性表中删除第一个
    \VAR{key} 项。若存在 \VAR{key},返回
    \retval{\T},否则返回 \retval{\NIL}。
  }

  \IT{(\SO*{PROGV} \VAR{symbols} \VAR{values} \PROGN{\VAR{form}})}
  {
    将 \VAR{values} 或 \NIL\ 动态绑定至 \VAR{symbols}
    并对 \VAR{form}s 求值。返回 \retval{\VAR{form}s
      的所有值}。
  }

  \IT{(\xorGOO{%
      \SO*{LET}\\ 
      \SO*{LET\A}}{\}} (%
    \orGOO{\VAR{name}\\
      (\VAR{name }\Op{\VAR{value}\DF{\NIL}})}{\}^{\!\!*}}) 
    \OPn{(\kwd{declare} \OPn{\NEV{\VAR{decl}}})} 
    \PROGN{\VAR{form}})}
  {
    平行/顺序将 \VAR{values} 词法绑定至 \VAR{name}s
    并对 \VAR{form}s 求值。返回
    \retval{\VAR{form}s 的所有值}。
  }

  \IT{(\MC*{MULTIPLE-VALUE-BIND} (\OPn{\NEV{\VAR{var}}}) \VAR{values-form}
    \OPn{(\kwd{declare} \OPn{\NEV{\VAR{decl}}})} \PROGN{\VAR{body-form}})}
  {
    将 \VAR{values-form} 的返回所有值词法绑定至 \VAR{var}s
    并对 \VAR{body-form}s 求值。返回 \retval{\VAR{body-form}s
      的所有值}。
  }

  \IT{(\MC*{DESTRUCTURING-BIND} \VAR{destruct-$\lambda$} \VAR{bar}
    \OPn{(\kwd{declare}
      \OPn{\NEV{\VAR{decl}}})} \PROGN{\VAR{form}})}
  {
    将树 \VAR{bar} 中元素绑定至树 \VAR{destruct-$\lambda$}
    中的相应位置并对 \VAR{form}s 求
    \retval{值}。\VAR{destruct-$\lambda$} 类似不带
    \kwd{\&environment} 子句的
    \VAR{macro-$\lambda$}(第 \ref{section:宏}
    节)。
  }

\end{LIST}


%%%%%%%%%%%%%%%%%%%%%%%%%%%%%%%%%%%%%%%%%%%%%%%%%%
\subsection{函数}
%%%%%%%%%%%%%%%%%%%%%%%%%%%%%%%%%%%%%%%%%%%%%%%%%%
\label{section:函数}

\begin{flushleft}
  下文中,一般 lambda 列表(\OPn{\VAR{ord-$\lambda$}})的形式为\\
  % \VAR{var} referenced from DEFSTRUCT
  (\OPn{\VAR{var}}
  \OP{\kwd{\&optional} \xorGOO{%
      \VAR{var}\\
      (\VAR{var } \OP{\VAR{init}\DF{\NIL}\text{ } \Op{\VAR{supplied-p}}})}{\}^{\!\!*}}}
  \penalty-5
  \Op{\kwd{\&rest} \VAR{var}}
  \penalty-5
  \OP{\kwd{\&key} \xorGOO{%
        \VAR{var}\\
        (\xorGOO{%
          \VAR{var}\\
          (\kwd{:}\VAR{key } \VAR{var})}{\}}
      \text{ }\OP{\VAR{init}\DF{\NIL}\text{ }\Op{\VAR{supplied-p}}})}{\}^{\!\!*}}
  \penalty-5
    \Op{\kwd{\&allow-other-keys}}}
  \penalty-5
  \OP{\kwd{\&aux} \xorGOO{%
      \VAR{var}\\
      (\VAR{var } \Op{\VAR{init}\DF{\NIL}})}{\}^{\!\!*}}}).
\end{flushleft}
  若存在相应参数,则 \VAR{supplied-p} 为
  \T。\VAR{init} 形式可引用任何位居其左的
  \VAR{init} 及 \VAR{supplied-p}。

\begin{LIST}{1cm}

  \IT{(\xorGOO{\MC*{DEFUN }\xorGOO{\VAR{foo }(\VAR{\OPn{ord-$\lambda$}})\\
      (\kwd{setf} \VAR{ foo})\text{ }(\VAR{new-value } \VAR{\OPn{ord-$\lambda$}})}{.}\\ 
      \MC*{LAMBDA }(\VAR{\OPn{ord-$\lambda$}})}{\}}
    \OPn{(\kwd{declare} \OPn{\NEV{\VAR{decl}}})}
    \Op{\NEV{\VAR{doc}}} 
    \PROGN{\VAR{form}})}
  {
    定义名为 \retval{\VAR{foo}} 或 \retval{(\kwd{setf}
      \VAR{foo})} 或匿名 \retval{函数},其中
    \VAR{form}s 以 \VAR{ord-$\lambda$}s 解析。对 \MC{defun},\VAR{form}s
    包含于名为 \VAR{foo} 的隐式 \SO{block} 中。
  }

  \IT{(\xorGOO{\SO*{FLET}\\
      \SO*{LABELS}}{\}}
    (\OPn{(\xorGOO{\VAR{foo }(\OPn{\VAR{ord-$\lambda$}})\\
        (\kwd{setf} \VAR{ foo})\text{ }(\VAR{new-value }\OPn{\VAR{ord-$\lambda$}})}{\}}
      \OPn{(\kwd{declare} \OPn{\NEV{\VAR{local-decl}}})}
      \Op{\NEV{\VAR{doc}}}
      \PROGN{\VAR{local-form}})}) \OPn{(\kwd{declare} \OPn{\NEV{\VAR{decl}}})}
    \PROGN{\VAR{form}})}
  {
    定义局部函数 \VAR{foo}
    并对 \VAR{form}s 求值。将遮蔽同名的
    全局函数。每个 \VAR{foo} 也是包含其所对应
    \OPn{\VAR{local-form}} 隐式 \SO{block} 的名称。仅对
    \SO{LABELS},函数 \VAR{foo} 在
    \VAR{local-forms} 内部可见。返回 \retval{\VAR{form}s 的所有值}。
  }

  \IT{(\SO*{FUNCTION} \xorGOO{%
      \VAR{foo}\\
      (\MC{lambda } \OPn{\VAR{form}})}{\}})}
  {
    返回最内的名为 \VAR{foo} 的词法 \retval{函数}
    或词法闭包中的 \retval{\MC{lambda} 表达式}。
  }

  \IT{(\FU*{APPLY} \xorGOO{\VAR{function}\\
      (\kwd{setf } \VAR{function})}{\}} \OPn{\VAR{arg}} \VAR{args})}
  {
    用 \VAR{arg}s 及 \VAR{args} 的列表元素调用
    \retval{\VAR{function} 的所有值}。若 \VAR{function} 为
    \FU{aref}、\FU{bit} 或 \FU{sbit},则可 \kwd{setf}。
  }

  \IT{(\FU*{FUNCALL} \VAR{function} \OPn{arg})}
  {
    用 \VAR{arg}s 调用 \retval{\VAR{function} 的所有值}。
  }

  \IT{(\SO*{MULTIPLE-VALUE-CALL} \VAR{function} \OPn{\VAR{form}})}
  {
    用每个 \VAR{form} 的所有值作为参数调用
    \VAR{function}。返回 \retval{\VAR{function} 的返回所有值}。
  }

  \IT{(\FU*{VALUES-LIST} \VAR{list})}
  {
    返回 \retval{\VAR{list} 的元素}。
  }

  \IT{(\FU*{VALUES} \OPn{\VAR{foo}})}
  {
  将 \VAR{foo}s 的 \retval{主值}
  作为多值返回。可 \kwd{setf}。
  }

  \IT{(\FU*{MULTIPLE-VALUE-LIST} \VAR{form})}
  {
    \retval{\VAR{form} 所有值列表}。
  }

  \IT{(\MC*{NTH-VALUE} \VAR{n} \VAR{form})}
  {
  \VAR{form} 的零索引 \retval{第\ \VAR{n} 返回值}。
  }

  \IT{(\FU*{COMPLEMENT} \VAR{function})}
  {
  返回与 \VAR{function} 参数相同、副作用相同但真值互补的
  \retval{新函数}。
  }

  \IT{(\FU*{CONSTANTLY} \VAR{foo})}
  {
    接受任意数量参数、返回 \VAR{foo} 的 \retval{函数}。
  }

  \IT{(\FU*{IDENTITY} \VAR{foo})}
  {
  返回 \retval{\VAR{foo}}。
  }

  \IT{(\FU*{FUNCTION-LAMBDA-EXPRESSION} \VAR{function})}
  {
  若可用,返回 \VAR{function} 的 \retval{lambda 表达式},\retvalii{\NIL}
  若函数定义于环境中而未绑定,及
  \VAR{function} 的 \retvaliii{名称}。
  }

  \IT{(\FU*{FDEFINITION} \xorGOO{\VAR{foo}\\
      (\kwd{setf } \VAR{foo})}{\}})}
  {
  全局函数 \VAR{foo} 的 \retval{定义}。可 \kwd{setf}。
  }

  \IT{(\FU*{FMAKUNBOUND} \VAR{foo})}
  {
  移除全局函数或宏定义 \retval{foo}。
  }

  \IT{\arrGOO{\CNS*{CALL-ARGUMENTS-LIMIT}\\
      \CNS*{LAMBDA-PARAMETERS-LIMIT}}{.}}
  {
  函数参数或 lambda
  列表参数数目的上界,$\geq50$。
  }

  \IT{\CNS*{MULTIPLE-VALUES-LIMIT}}
  {
  多值数目的上界,$\geq20$。
  }

\end{LIST}


%%%%%%%%%%%%%%%%%%%%%%%%%%%%%%%%%%%%%%%%%%%%%%%%%%
\subsection{宏}
%%%%%%%%%%%%%%%%%%%%%%%%%%%%%%%%%%%%%%%%%%%%%%%%%%
\label{section:宏}

\begin{flushleft}
  下文中,宏 lambda
  列表(\OPn{\VAR{macro-$\lambda$}})为以下两种形式中的一种
  \penalty-5
  (\Op{\kwd{\&whole} \VAR{var}}
  \penalty-5
  \Op{\VAR{E}}
  \penalty-5
  \xorGOO{%
    \VAR{var}\\
    (\OPn{\VAR{macro-$\lambda$}})}{\}^{\!\!*}}
  \penalty-5
  \Op{\VAR{E}}
  \penalty-5
  \Op{\kwd{\&optional} 
    \xorGOO{%
      \VAR{var}\\
      (\xorGOO{%
        \VAR{var}\\
        (\OPn{\VAR{macro-$\lambda$}})}{\}}\text{ }
      \OP{\VAR{init}\DF{\NIL}\text{ } \Op{\VAR{supplied-p}}})}{\}^{\!\!*}}}
  \penalty-5
  \Op{\VAR{E}}
  \penalty-5
  \Op{%
    \xorGOO{%
      \kwd{\&rest}\\
      \kwd{\&body}}{\}}
    \xorGOO{%
      \VAR{rest-var}\\
      (\OPn{\VAR{macro-$\lambda$}})}{\}}}
  \penalty-5
  \Op{\VAR{E}}
  \penalty-5
  \OP{\kwd{\&key} 
    \xorGOO{%
      \VAR{var}\\
      (\xorGOO{%
        \VAR{var}\\
        (\kwd{:}\VAR{key } \xorGOO{%
          \VAR{var}\\
          (\OPn{\VAR{macro-$\lambda$}})}{\}})}{\}}
      \text{ }\OP{\VAR{init}\DF{\NIL}\text{ } \Op{\VAR{supplied-p}}})}{\}^{\!\!*}}
    \Op{\VAR{E}}
    \Op{\kwd{\&allow-other-keys}}}
  \penalty-5
  \Op{\kwd{\&aux} 
    \xorGOO{%
      \VAR{var}\\
      (\VAR{var } \Op{\VAR{init}\DF{\NIL}})}{\}^{\!\!*}}}
  \penalty-5
  \Op{\VAR{E}})
  \\
  或
  \\
  \penalty-5
  (\Op{\kwd{\&whole}
      \VAR{var}}
  \penalty-5
  \Op{\VAR{E}}
  \penalty-5
  \xorGOO{%
    \VAR{var}\\
    (\OPn{\VAR{macro-$\lambda$}})}{\}^{\!\!*}}
  \penalty-5
  \Op{\VAR{E}}
  \penalty-5
  \Op{\kwd{\&optional} 
    \xorGOO{%
      \VAR{var}\\
      (\xorGOO{%
        \VAR{var}\\
        (\OPn{\VAR{macro-$\lambda$}})}{\}}\text{ }
      \OP{\VAR{init}\DF{\NIL}\text{ } \Op{\VAR{supplied-p}}})}{\}^{\!\!*}}}
  \penalty-5
  \Op{\VAR{E}}
  \kwd{.}
  \VAR{rest-var}).
  \penalty-5
\end{flushleft}
  顶层 \Op{\VAR{E}} 可用 \kwd{\&environment} \VAR{var}
  替换。若存在相应参数,则 \VAR{supplied-p} 为
  \T。\VAR{init} 形式可引用任何位居其左的
  \VAR{init} 及 \VAR{supplied-p}。

\begin{LIST}{1cm}

  \IT{(\xorGOO{\MC*{DEFMACRO}\\
      \FU*{DEFINE-COMPILER-MACRO}}{\}}
    \xorGOO{\VAR{foo}\\
    (\kwd{setf } \VAR{foo})}{\}}
    (\OPn{\VAR{macro-$\lambda$}})
    \OPn{(\kwd{declare} \OPn{\NEV{\VAR{decl}}})} \Op{\NEV{\VAR{doc}}}
    \PROGN{\VAR{form}})}
  {
  定义宏 \retval{\VAR{foo}},求值时以 (\VAR{foo}
  \VAR{tree}) 将展开的、与 \VAR{tree} 形状相对应的
  \VAR{macro-$\lambda$} \VAR{form}s 应用至参数
  \VAR{tree}。\VAR{form}s 包含于名为 \VAR{foo}
  的隐式 \SO{block} 中。
  }

  \IT{(\MC*{DEFINE-SYMBOL-MACRO} \VAR{foo} \VAR{form})}
  {
  定义符号宏 \retval{\VAR{foo}},求值时对已展开的
  \VAR{form} 求值。
  }

  \IT{(\SO*{MACROLET} (\OPn{(\VAR{foo} (\OPn{\VAR{macro-$\lambda$}})
      \OPn{(\kwd{declare} \OPn{\NEV{\VAR{local-decl}}})} 
      \Op{\NEV{\VAR{doc}}} \PROGN{\VAR{macro-form}})}) \OPn{(\kwd{declare}
      \OPn{\NEV{\VAR{decl}}})} \PROGN{\VAR{form}})}
  {
  局部定义互相不可见、包含于同名隐式
  \SO{block}s 中的宏 \VAR{foo} 并对
  \retval{\VAR{form}s} 求值。
  }

  \IT{(\SO*{SYMBOL-MACROLET} (\OPn{(\VAR{foo}
      \VAR{expansion-form})}) \OPn{(\kwd{declare} \OPn{\NEV{\VAR{decl}}})}
    \PROGN{\VAR{form}})}
  {
  局部定义符号宏 \VAR{foo} 并对
  \retval{\VAR{form}s} 求值。
  }

  \IT{(\MC*{DEFSETF} \NEV{\VAR{function}} \xorGOO{%
      \NEV{\VAR{updater}} \text{ } \Op{\NEV{\VAR{doc}}}\\
      (\OPn{\VAR{setf-$\lambda$}}) \text{ } (\OPn{\VAR{s-var}}) \text{ } 
        \OPn{(\kwd{declare } \OPn{\NEV{\VAR{decl}}})}\text{ }
        \Op{\NEV{\VAR{doc}}} \text{ } \PROGN{\VAR{form}}}{\}})
    \penalty-5
  其中 defsetf lambda 列表(\OPn{\VAR{setf-$\lambda$}})的形式为
  \penalty-5
  (\OPn{\VAR{var}}
  \OP{\kwd{\&optional} \xorGOO{%
      \VAR{var}\\
      (\VAR{var } \OP{\VAR{init}\DF{\NIL}\text{ } \Op{\VAR{supplied-p}}})}{\}^{\!\!*}}}
  \penalty-5
  \Op{\kwd{\&rest} \VAR{var}}
  \penalty-5
  \OP{\kwd{\&key} \xorGOO{%
      \VAR{var}\\
      (\xorGOO{%
        \VAR{var}\\
        (\kwd{:}\VAR{key } \VAR{var})}{\}}
      \text{ }\OP{\VAR{init}\DF{\NIL}\text{ }\Op{\VAR{supplied-p}}})}{\}^{\!\!*}}
    \penalty-5
    \Op{\kwd{\&allow-other-keys}}}
  \penalty-5
  \OP{\kwd{\&environment} \VAR{var}}%
  )
  }
  {
    指定如何通过
    \retval{\VAR{function}} \kwd{setf}
    位置。\EM{短形式:}用 (\VAR{updater} \OPn{\VAR{arg}} \VAR{value-form})
    替换 (\kwd{setf} (\VAR{function} \OPn{\VAR{arg}}) \VAR{value-form}),后者须返回
    \VAR{value-form}。\EM{长形式:}调用
    (\kwd{setf} (\VAR{function}
    \OPn{\VAR{arg}}) \VAR{value-form}) 时,\VAR{form}s
    须展开成设置位置的代码,其中 \VAR{setf-$\lambda$}
    和 \OPn{\VAR{s-var}} 分别指定 \VAR{function}
    的参数及将储存的值,并返回
    \OPn{\VAR{s-var}}的值。\VAR{form}s 包含于名为
    \VAR{function} 的隐式 \SO{block} 中。
  }

  \IT{(\MC*{DEFINE-SETF-EXPANDER} \VAR{function}
    (\OPn{\VAR{macro-$\lambda$}})
    \OPn{(\kwd{declare} \OPn{\NEV{\VAR{decl}}})} \Op{\NEV{\VAR{doc}}}
    \PROGN{\VAR{form}})}
  {
  指定如何通过 \retval{\VAR{function}}
  \kwd{setf} 位置。调用 (\kwd{setf} (\VAR{function}
    \OPn{\VAR{arg}}) \VAR{value-form}) 时,\OPn{\VAR{form}}
    须展开为代码,其返回
    \VAR{arg-vars}、\VAR{args}、\VAR{newval-vars}、\VAR{set-form} 及由
    \FU{GET-SETF-EXPANSION} 所指定的 \VAR{get-form},宏 lambda 列表
    \OPn{\VAR{macro-$\lambda$}} 的元素绑定至对应的
    \VAR{arg}s 上。\VAR{form}s 包含于名为
    \VAR{function} 的隐式 \SO{block} 中。
  }

  \IT{(\FU*{GET-SETF-EXPANSION} \VAR{place} \Op{\VAR{environment}\DF{\NIL}})}
  {
    返回临时变量 \retval{\VAR{arg-vars}}
    的列表,与
    \VAR{place} 对应的
    \retvalii{\VAR{args}},对应新值的临时变量的列表
    \retvaliii{\VAR{\VAR{newval-vars}}},\retvaln{4}{\VAR{set-form}},\VAR{arg-vars}、\VAR{newval-vars}
    \kwd{setf} 及读取 \VAR{place} 方式
    指定的 \retvaln{5}{\VAR{get-form}}。
  }

  \IT{(\MC*{DEFINE-MODIFY-MACRO} \VAR{foo} 
    (\OP{\kwd{\&optional} \xorGOO{%
        \VAR{var}\\
        (\VAR{var } \OP{\VAR{init}\DF{\NIL}\text{ }
          \Op{\VAR{supplied-p}}})}{\}^{\!\!*}}}
    \Op{\kwd{\&rest} \VAR{var}}) 
    \VAR{function} \Op{\NEV{\VAR{doc}}})}
  {
  定义可修改位置的宏 \retval{\VAR{foo}}。调用
  (\VAR{foo} \VAR{place} \OPn{\VAR{arg}}) 时,将
  \VAR{function} 应用至 \VAR{place} 及
  \VAR{arg}s,其值将储存至 \VAR{place} 并返回。
  } 

  \IT{\CNS*{LAMBDA-LIST-KEYWORDS}}
  {宏 lambda 列表的关键字列表。至少包含:
    }
    \begin{LIST}{.5cm}

      \IT{\kwd*{\&whole} \VAR{var}}
      {将 \VAR{var} 绑定至整个宏调用形式。}

      \IT{\kwd*{\&optional} \OPn{\VAR{var}}}
      {若存在,将 \VAR{var}s 绑定至对应参数。}

      \IT{\Goo{%
          \kwd*{\&rest}\XOR
          \kwd*{\&body}} \VAR{var}}
      {将 \VAR{var} 绑定至剩余参数列表。}

      \IT{\kwd*{\&key} \OPn{\VAR{var}}}
      {将 \VAR{var}s 绑定至对应关键字参数。}

      \IT{\kwd*{\&allow-other-keys}}
      {
      禁用关键字参数检查。调用者可以使用
      \kwd*{:allow-other-keys}~\T。
      }
      
      \IT{\kwd*{\&environment} \VAR{var}}
      {将 \VAR{var} 绑定至词法编译环境。}

      \IT{\kwd*{\&aux} \OPn{\VAR{var}}}
      {在 \SO{let\A} 内绑定 \VAR{var}s。}
          
    \end{LIST}
\end{LIST}


%%%%%%%%%%%%%%%%%%%%%%%%%%%%%%%%%%%%%%%%%%%%%%%%%%
\subsection{控制流}
%%%%%%%%%%%%%%%%%%%%%%%%%%%%%%%%%%%%%%%%%%%%%%%%%%
\begin{LIST}{1cm}

  \IT{(\SO*{IF} \VAR{test} \VAR{then} \Op{\VAR{else}\DF{\NIL}})}
  {
    若 \VAR{test} 返回 \T,则返回 \retval{\VAR{then}}
    的所有值,否则返回 \retval{\VAR{else}} 的所有值。
  }

  \IT{(\MC*{COND} \OPn{(\VAR{test} \PROGN{\VAR{then}}\DF{\VAR{test}})})}
  {
  返回第一个 \VAR{test} 返回 \T\ 的
  \OPn{\VAR{then}} \retval{所有值};若所有
  \VAR{test}s 均返回 \NIL,返回 \NIL。
  }

  \IT{(\xorGOO{\MC*{WHEN}\\
      \MC*{UNLESS}}{\}} \VAR{test}
    \PROGN{\VAR{foo}})}
  {
    若 \VAR{test} 返回 \T\ 或 \NIL,对 \VAR{foo}s
    求值并返回 \retval{其所有值},否则返回 \retval{\NIL}。
  }

  \IT{(\MC*{CASE} \VAR{test} \OPn{(\xorGOO{(\OPn{\NEV{\VAR{key}}})\\
        \NEV{\VAR{key}}}{\}} \PROGN{\VAR{foo}})}
    \OP{(\xorGOO{\kwd*{OTHERWISE}\\
      \T}{\}} \PROGN{\VAR{bar}})\DF{\NIL}})}
  {
    返回第一个其 \VAR{key}s 中有一 \kwd{eql} \VAR{test} 的
    \retval{\OPn{\VAR{foo}} 值}。若无匹配 \VAR{key},返回 \retval{\VAR{bar}s
      的值}。
    % Keep \OPn{\VAR{foo}} instead of the otherwise preferable
    % \VAR{foo}s because we're talking about one set of foos out
    % of several.
  }

  \IT{(\xorGOO{\MC*{ECASE}\\
      \MC*{CCASE}}{\}} \VAR{test}
    \OPn{(\xorGOO{(\OPn{\NEV{\VAR{key}}})\\
        \NEV{\VAR{key}}}{\}} \PROGN{\VAR{foo}})})}
  {
    返回第一个其 \VAR{key}s 中有一 \kwd{eql} \VAR{test} 的
    \retval{\OPn{\VAR{foo}} 值}。若无匹配
    \VAR{key},产生不可纠正/可纠正的
    \kwd{type-error}。
    % Keep \OPn{\VAR{foo}} instead of the otherwise preferable
    % \VAR{foo}s because we're talking about one set of foos out
    % of several.
  }

  \IT{(\MC*{AND} \OPn{\VAR{form}}\DF{\T})}
  {
    从左至右求值 \VAR{form}s。若其一
    \VAR{form} 值为 \NIL,立即返回 \retval{\NIL},否则返回
    \retval{最后一个\ \VAR{form} 的值}。
  }

  \IT{(\MC*{OR} \OPn{\VAR{form}}\DF{\NIL})}
  {
    从左至右求值 \VAR{form}s。若其一
    form 值非 \NIL,立即返回 \retval{其主值},或返回
    最后一个 \VAR{form} 的 \retval{所有值}。若无
    \VAR{form} 返回 \T,返回 \retval{\NIL}。
  }

  \IT{(\SO*{PROGN} \OPn{\VAR{form}}\DF{\NIL})}
  {\label{:progn}
    顺序求值 \VAR{form}s。返回
    \retval{最后一个\ \VAR{form} 的所有值}。
  }

  \IT{\arrGOO{%
      (\SO*{MULTIPLE-VALUE-PROG1} \VAR{ form-r} \OPn{\VAR{ form}})\\
      (\MC*{PROG1} \VAR{ form-r} \OPn{\VAR{ form}})\\
      (\MC*{PROG2} \VAR{ form-a} \VAR{ form-r} \OPn{\VAR{ form}})}{.}}
  {
  顺序求值 forms。返回 \VAR{form-r}
  的 \retval{所有值/主值}。
  }

  \IT{(\xorGOO{\MC*{PROG}\\
      \MC*{PROG\A}}{\}}
    (\orGOO{%
      \VAR{name}\\
      (\VAR{name } \Op{\VAR{value}\DF{\NIL}})}{\}^{\!\!*}})
    \OPn{(\kwd{declare} \OPn{\NEV{\VAR{decl}}})} 
    \xorGOO{\NEV{\VAR{tag}}\\\VAR{form}}{\}^{\!\!*}})}
  {
    将 \VAR{name}s (平行或顺序)词法绑定至 \VAR{value}s
    并对类 \SO{TAGBODY} 体求值。返回
    \retval{\NIL} 或显式 \retval{\MC{return} 值}。整个 form
    包含于名为 \NIL 的隐式 \SO{block} 中。
  }

  \IT{(\SO*{UNWIND-PROTECT} \VAR{protected} \OPn{\VAR{cleanup}})}
  {
    求值 \VAR{protected},无论控制流如何离开
    \VAR{protected},执行 \VAR{cleanup}。返回 \retval{\VAR{protected}
      的值}。
  }

  \IT{(\SO*{BLOCK} \VAR{name} \PROGN{\VAR{form}})}
  { 
    在词法环境中对 \VAR{form}s 求值,返回
    \retval{其所有值},除非被
    \SO{RETURN-FROM} 中断。
  }

  \IT{\arrGOO{(\SO*{RETURN-FROM } \VAR{foo } \Op{\VAR{result}\DF{\NIL}})\\
      (\MC*{RETURN } \Op{\VAR{result}\DF{\NIL}})}{.}}
  {
    从最近名为 \VAR{foo}/\NIL 的 \SO{block}
    中返回 \VAR{result} 的所有值。
  }

  \IT{(\SO*{TAGBODY} \Goos{\NEV{\VAR{tag}}\XOR\VAR{form}})}
  {
    在词法环境中对 \VAR{form}s 求值。\VAR{tag}s
    (符号或整数)有词法范围和动态范围,并且为
    \SO{GO} 的目标。返回 \retval{\NIL}。
  }

  \IT{(\SO*{GO} \NEV{\VAR{tag}})}
  {
    尽可能在最内的 \SO{tagbody} 跳转至
    \FU{eql} \VAR{tag} 的标签处。
  }

  \IT{(\SO*{CATCH} \VAR{tag} \PROGN{\VAR{form}})}
  {
    求值 \VAR{form}s 并返回 \retval{其所有值},除非被
    \SO{THROW} 中断。
  }

  \IT{(\SO*{THROW} \VAR{tag} \VAR{form})}
  {
    在有 \FU{eq} \VAR{tag}
    的标签的最内动态 \SO{CATCH} 中返回
    \VAR{form} 的所有值。
  }

  \IT{(\FU*{SLEEP} \VAR{n})}
  {
    等待 \VAR{n} 秒,返回 \retval{\NIL}。
  }

\end{LIST}


%%%%%%%%%%%%%%%%%%%%%%%%%%%%%%%%%%%%%%%%%%%%%%%%%%
\subsection{迭代}
%%%%%%%%%%%%%%%%%%%%%%%%%%%%%%%%%%%%%%%%%%%%%%%%%%

\begin{LIST}{1cm}
  
  \IT{(\xorGOO{\MC*{DO}\\
      \MC*{DO\A}}{\}}
    (\xorGOO{%
      \VAR{var}\\
      (\VAR{var } \OP{ \VAR{start } \Op{\VAR{step}}})}{\}^{\!\!*}})
    (\VAR{stop} \PROGN{\VAR{result}})
    \OPn{(\kwd{declare} \OPn{\NEV{\VAR{decl}}})} 
    \xorGOO{\NEV{\VAR{tag}}\\\VAR{form}}{\}^{\!\!*}})}
  {
    平行/顺序将 \VAR{var}s 依次绑定至先
    \VAR{start} 后 \VAR{step}
    形式的值并对类 \SO{TAGBODY} 体求值。当
    \VAR{stop} 为 \T 时停止迭代并返回
    \retval{\OPn{\VAR{result}} 的所有值}。整个
    form 包含于名为 \NIL 的隐式 \SO{block} 中。
  }

  \IT{(\MC*{DOTIMES} (\VAR{var} \VAR{i} \Op{\VAR{result}\DF{\NIL}})
    \OPn{(\kwd{declare} \OPn{\NEV{\VAR{decl}}})}
    \Goos{\NEV{\VAR{tag}}\XOR\VAR{form}})}
  {
    将 \VAR{var}s 依次绑定至 0 至 $i - 1$
    的整数。在求值 \retval{\VAR{result}}
    时,\VAR{var} 为 \VAR{i}。整个
    form 包含于名为 \NIL 的隐式 \SO{block} 中。
  }

  \IT{(\MC*{DOLIST }(\VAR{var} \VAR{list} \Op{\VAR{result}\DF{\NIL}})
    \OPn{(\kwd{declare} \OPn{\NEV{\VAR{decl}}})}
    \Goos{\NEV{\VAR{tag}}\XOR\VAR{form}})}
  {
    将 \VAR{var}s 依次绑定至 \VAR{list}
    的元素。在求值 \retval{\VAR{result}}
    时,\VAR{var} 为 \NIL。整个
    form 包含于名为 \NIL 的隐式 \SO{block} 中。
  }


\end{LIST}

%%%%%%%%%%%%%%%%%%%%%%%%%%%%%%%%%%%%%%%%%%%%%%%%%%
\subsection{Loop 宏}
%%%%%%%%%%%%%%%%%%%%%%%%%%%%%%%%%%%%%%%%%%%%%%%%%%
\label{section:Loop 宏}

\begin{LIST}{1cm}

  \IT{(\MC*{LOOP} \OPn{\VAR{form}})} 
  {
    \EM{简单 Loop。}若 \VAR{form}s 未包含任何原子 Loop
    宏关键字,在名为 \NIL 的隐式 \SO{block}
    中无限对其求值。
  }

  \IT{(\MC*{LOOP} \OPn{\VAR{clause}})}
  {
    \EM{Loop 宏。}Loop 宏关键字见下及图
    \ref{loop-overview}。
  }

  \begin{LIST}{.5cm}
    
    \IT{\LKWD*{named} \VAR{n}\DF{\NIL}}
    {
      为 \MC{loop} 的隐式 \SO{block} 指定名称。
    }

    \IT{\Goop{\LKWD*{with}
        \xorGOO{%
          \VAR{var-s}\\
          (\OPn{\VAR{var-s}})}{\}}
        \Op{\VAR{d-type}}
        \Op{\LKWD*{=} \VAR{foo}}}
      \Goos{\LKWD*{and}
        \xorGOO{%
          \VAR{var-p}\\
          (\OPn{\VAR{var-p}})}{\}}
        \Op{\VAR{d-type}}
        \Op{\LKWD*{=} \VAR{bar}}}
      \penalty-10
      其中解构类型说明符 \VAR{d-type} 的形式为
      \penalty-5
      \GOO{\kwd{fixnum}\XOR
      \kwd{float}\XOR
      \T\XOR
      \NIL\XOR
      \GOo{%
      \LKWD*{of-type}
      \xorGOO{\VAR{type}\\
      (\OPn{\VAR{type}})}{\}}}}
    }
    {
      顺序初始化局部变量(树)
      \VAR{var-s},平行初始化 \VAR{var-p}。
    }

    \begin{figure}%
      \begin{center}%
        \begin{sideways}%
          (%  
          \arraycolsep0pt
          \(
          \text{\kwd{loop}}\text{ }
          \Op{
            \text{\LKWD{named} \VAR{n}\DF{\NIL}}
          }
          \left\{
            \begin{array}{l}
              \text{\LKWD{with}
                \xorGOO{%
                  \VAR{var}\\
                  (\OPn{\VAR{var}})}{\}}
                \Op{\VAR{d-type}}
                \Op{\LKWD{=} \VAR{foo}}}
              \text{ }
              \{\text{\LKWD{and}
                \xorGOO{%
                  \VAR{var}\\
                  (\OPn{\VAR{var}})}{\}}
                \Op{\VAR{d-type}}
                \Op{\LKWD{=} \VAR{bar}}}\}^{*} \\ \\[-2.4ex]
              \left.\!
                \begin{array}{l}
                  \text{\LKWD{for}}  \\
                  \text{\LKWD{as}} 
                \end{array}\right\}
              \boxed{
                \text{\xorGOO{%
                    \VAR{var}\\
                    (\OPn{\VAR{var}})}{\}}}%
                \Op{\VAR{d-type}}
                \left\{
                  \begin{array}{l}
                    \left.\!
                      \begin{array}{l}
                        \OP{
                          \left\{
                            \begin{array}{l}
                              \text{\LKWD{upfrom}}\\
                              \text{\LKWD{from}}
                            \end{array}
                          \right\}
                          \text{\VAR{start}\DF{\LIT{0}}}
                        }\text{ }
                        \OP{
                          \left\{
                            \begin{array}{l}
                              \text{\LKWD{upto}}\\
                              \text{\LKWD{to}}\\
                              \text{\LKWD{below}}
                            \end{array}
                          \right\}
                          \text{\VAR{form}}
                        }\\
                        \text{\LKWD{from} \VAR{start}}
                        \left\{
                          \begin{array}{l}
                            \text{\LKWD{downto}}\\
                            \text{\LKWD{above}}
                          \end{array}
                        \right\}              
                        \text{\VAR{form }}\\
                        \text{\LKWD{downfrom} \VAR{start}}\text{ }
                        \OP{
                          \left\{
                            \begin{array}{l}
                              \text{\LKWD{downto}}\\
                              \text{\LKWD{to}}\\
                              \text{\LKWD{above}}
                            \end{array}
                          \right\}            
                          \text{\VAR{form}}
                        }
                      \end{array}
                    \right\}
                    \Op{   
                      \text{\LKWD{by} \VAR{step}\DF{\LIT{1}}}
                    }\\
                    \left.\!
                      \begin{array}{l}
                        \text{\LKWD{in}} \\
                        \text{\LKWD{on}}
                      \end{array}
                    \right\}
                    \text{\VAR{list}}\text{ }
                    \Op{
                      \text{\LKWD{by} \VAR{function}\DF{\kwd{\#'cdr}}}
                    }\\
                    \text{\LKWD{=} \VAR{foo }} 
                    \Op{
                      \text{\LKWD{then} \VAR{bar}\DF{\VAR{foo}}}
                    }\\
                    \text{\LKWD{across} \VAR{vector}}\\
                    \text{\LKWD{being}}
                    \left\{
                      \begin{array}{l}
                        \text{\LKWD{the}}\\          
                        \text{\LKWD{each}}
                      \end{array}
                    \right\}%
                    \left\{
                      \begin{array}{{l}}
                        \left.\!
                          \begin{array}{l}
                            \text{\LKWD{hash-key\Op{s}}}
                            \left\{
                              \begin{array}{l}
                                \text{\LKWD{of}}\\
                                \text{\LKWD{in}}
                              \end{array}
                            \right\}
                            \text{\VAR{hash }} 
                            \Op{
                              \text{\LKWD{using} (\LKWD{hash-value} \VAR{v})}%
                            }\\
                            \text{\LKWD{hash-value\Op{s}}}
                            \left\{
                              \begin{array}{l}
                                \text{\LKWD{of}}\\
                                \text{\LKWD{in}}
                              \end{array}
                            \right\}
                            \text{\VAR{hash }}
                            \Op{
                              \text{\LKWD{using} (\LKWD{hash-key} \VAR{k})}%
                            }\\
                          \end{array}
                        \right.\\
                        \left.\!
                          \begin{array}{l}
                            \text{\LKWD{symbol\Op{s}}}\\
                            \text{\LKWD{present-symbol\Op{s}}}\\
                            \text{\LKWD{external-symbol\Op{s}}}
                          \end{array}
                        \right\}%
                        \OP{
                          \left\{
                            \begin{array}{l}
                              \text{\LKWD{of}}\\
                              \text{\LKWD{in}}
                            \end{array}
                          \right\} \text{\VAR{package}\DF{\V{\A package\A}}}%
                        }%
                      \end{array}\!\!%
                    \right.%
                  \end{array}%
                \right\}_{{}^{{}^{\displaystyle{\mathbb{F}_0}}}}
              }%boxed
              \left\{
                \text{\LKWD{and} }\boxed{\mathbb{F}_i\!}
              \right\}^{\!*}\\
              \\[-2.4ex]
              \boxed{\;\mathbb{T}_1\,}
              
            \end{array}
          \right\}^{\displaystyle{\!\!*}}
          % Bottleneck
          \left\{
            \begin{array}{l}
              \boxed{
                \left.\!
                  \begin{array}{l}
                    \text{\LKWD{do\Op{ing}}}
                    \text{ \VAR{form}}^{+}\\
                    \left.\!
                      \begin{array}{l}
                        \text{\LKWD{if}}\\
                        \text{\LKWD{when}}\\
                        \text{\LKWD{unless}}
                      \end{array}
                    \right\}%
                    \text{\VAR{test }}
                    \boxed{\mathbb{C}_i\!}\{\text{\LKWD{and}}\,
                    \boxed{\mathbb{C}_j\!}\}^{*}\Op{\text{\LKWD{else}}\,
                      \boxed{\mathbb{C}_k\!}\{\text{\LKWD{and}}\,
                      \boxed{\mathbb{C}_l\!}\}^{*}}
                    \Op{\text{\LKWD{end}}}\!\!\!\!\!\!\!\\
                    \text{\LKWD{return}}
                    \left\{
                      \begin{array}{l}
                        \text{\VAR{form}}\\
                        \text{\LKWD{it}}
                      \end{array}
                    \right.\\
                    \left.\!
                      \begin{array}{l}
                        \text{\LKWD{collect\Op{ing}}}\\
                        \text{\LKWD{append\Op{ing}}}\\
                        \text{\LKWD{nconc\Op{ing}}}
                      \end{array}
                    \right\}
                    \left\{
                      \begin{array}{l}
                        \text{\VAR{form}}\\
                        \text{\LKWD{it}}
                      \end{array}
                    \right\} \Op{\text{\LKWD{into }\VAR{list}}}\\
                    \left.\!
                      \begin{array}{l}
                        \text{\LKWD{count\Op{ing}}}\\
                        \text{\LKWD{sum\Op{ming}}}\\
                        \text{\LKWD{maximize}}\\
                        \text{\LKWD{maximizing}}\\
                        \text{\LKWD{minimize}}\\
                        \text{\LKWD{minimizing}}
                      \end{array}
                    \right\}
                    \left\{
                      \begin{array}{l}
                        \text{\VAR{form}}\\
                        \text{\LKWD{it}}
                      \end{array}
                    \right\} 
                    \Op{\text{\LKWD{into }\VAR{num}}}\text{ }
                    \Op{\text{\VAR{type}}}
                  \end{array}
                \right._{{}^{{}^{\!\!\!\!\displaystyle{\mathbb{C}_0}}}}
              }%boxed
              \\
              \\[-2.4ex]
              \boxed{
                \left.\!
                  \begin{array}{l}
                    \left.\!
                      \begin{array}{l}
                        \text{\LKWD{initially}} \\
                        \text{\LKWD{finally}} 
                      \end{array} 
                    \right\}
                    \text{\VAR{form}}^{+}\\
                    \text{\LKWD{repeat} \VAR{num}}\\
                    \left.\!
                      \begin{array}{l}
                        \text{\LKWD{while}}\\
                        \text{\LKWD{until}}\\
                        \text{\LKWD{always}}\\
                        \text{\LKWD{never}}\\
                        \text{\LKWD{thereis}}
                      \end{array}
                    \right\} 
                    \text{\VAR{test}}
                  \end{array}
                \right._{{}^{{}^{\displaystyle{\mathbb{T}_2}}}}
              }%boxed
            \end{array}
          \right\}^{\displaystyle{\!\!*}}
          \)%
          \!)%
        \end{sideways}%
      \end{center}%
      \vspace{-3em}
      \raisebox{0em}[0em][0em]{\parbox[b]{4cm}{\caption{\protect\raggedright Loop 宏一览。\label{loop-overview}}}}
    \end{figure}

    \IT{\GOop{\Goo{\LKWD*{for}\XOR\LKWD*{as}}
        \xorGOO{%
          \VAR{var-s}\\
          (\OPn{\VAR{var-s}})}{\}}
        \Op{\VAR{d-type}}} 
      \GOos{\LKWD*{and}
        \xorGOO{%
          \VAR{var-p}\\
          (\OPn{\VAR{var-p}})}{\}}
        \Op{\VAR{d-type}}}}
    {
    迭代控制语句开头。顺序初始化局部变量(树)
    \VAR{var-s},平行初始化 \VAR{var-p}。解构类型说明符
    \VAR{d-type} 类似
    \LKWD{with}。
    }

    \begin{LIST}{.5cm}

      \IT{\Goo{\LKWD*{upfrom}\XOR\LKWD*{from}\XOR\LKWD*{downfrom}}
        \VAR{start}}
      {
        以 \VAR{start} 为起始值。
      }

      \IT{\Goo{\LKWD*{upto}\XOR\LKWD*{downto}\XOR\LKWD*{to}\XOR\LKWD*{below}\XOR\LKWD*{above}}
        \VAR{form}}
      {
        以 \VAR{form} 为终止值。
      }

      \IT{\Goo{\LKWD*{in}\XOR\LKWD*{on}} \VAR{list}}
      {
        顺序将 \VAR{var} 绑定至
        \VAR{list} 的元素/尾部。
      }

      \IT{\LKWD*{by} \Goo{\VAR{step}\DF{\LIT{1}}\XOR\VAR{function}\DF{\kwd{\#'cdr}}}}
      {
        指定(正)递减或递增量,或返回列表下一部分的单参
        \VAR{function}。
      }

      \IT{\LKWD*{=} \VAR{foo} \Op{\LKWD*{then}
          \VAR{bar}\DF{\VAR{foo}}}}
      {
        将 \VAR{var} 先绑定至 \VAR{foo} 后绑定至
        \VAR{bar}。
      }

      \IT{\LKWD*{across} \VAR{vector}}
      {
        顺序将 \VAR{var} 绑定至 \VAR{vector} 的元素。
      }

      \IT{\LKWD*{being} \Goo{\LKWD*{the}\XOR\LKWD*{each}}}
      {
        遍历散列表或包。
      }

      \begin{LIST}{.5cm}

        \IT{\Goo{\LKWD*{hash-key}\XOR\LKWD*{hash-keys}}
          \Goo{\LKWD*{of}\XOR\LKWD*{in}} \VAR{hash-table}
          \Op{\LKWD*{using} (\LKWD*{hash-value} \VAR{value})}}
        {
          将 \VAR{var} 依次绑定至 \VAR{hash-table}
          的键,将 \VAR{value} 绑定至对应的值。
        }

        \IT{\Goo{\LKWD*{hash-value}\XOR\LKWD*{hash-values}}
          \Goo{\LKWD*{of}\XOR\LKWD*{in}} \VAR{hash-table}
          \Op{\LKWD*{using} (\LKWD*{hash-key} \VAR{key})}}
        {
          将 \VAR{var} 依次绑定至 \VAR{hash-table}
          的值,将 \VAR{key} 绑定至对应的键。
        }

        \IT{\Goo{\LKWD*{symbol}\XOR\LKWD*{symbols}\XOR\LKWD*{present-symbol}\XOR\LKWD*{present-symbols}\XOR\LKWD*{external-symbol}\XOR\LKWD*{external-symbols}}
          \Op{\Goo{\LKWD*{of}\XOR\LKWD*{in}}
            \VAR{package}\DF{\V{\A package\A}}}}
        {
          将 \VAR{var} 依次绑定至 \VAR{package}
          的可访问符号,或当前符号,或外部符号。
        }

      \end{LIST}
    \end{LIST}
    
    \IT{\Goo{\LKWD*{do}\XOR\LKWD*{doing}} \RP{\VAR{form}}}
    {
      每次迭代中求值 \VAR{form}s。
    }

    \IT{\Goo{\LKWD*{if}\XOR\LKWD*{when}\XOR\LKWD*{unless}} \VAR{ test}
      \VAR{i-clause} \Goos{\LKWD*{and} 
        \VAR{j-clause}} \Op{\LKWD*{else} \VAR{k-clause} \Goos{\LKWD*{and} 
          \VAR{l-clause}}} \Op{\LKWD*{end}}} 
    {
      若 \VAR{test} 返回 \T、\T、\NIL,求值
      \VAR{i-clause} 及 \VAR{j-clause}s,否则求值
      \VAR{k-clause} 及 \VAR{l-clause}s。
    }

    \begin{LIST}{.5cm}
      \IT{\LKWD*{it}}
      {
        在 \VAR{i-clause} 或 \VAR{k-clause}
        中:\retval{\VAR{test} 的值}。
      }
    \end{LIST}

    \IT{\LKWD*{return} \Goo{\VAR{form}\XOR\LKWD*{it}}}
    {
      跳过 \LKWD{finally} 部分,立即返回
      \VAR{form} 的所有值或 \LKWD{it}。
    }

    \IT{\Goo{\LKWD*{collect}\XOR\LKWD*{collecting}}
      \Goo{\VAR{form}\XOR\LKWD*{it}} \Op{\LKWD*{into} \VAR{list}}}
    {
      收集 \VAR{form} 的值或 \LKWD{it} 至
      \VAR{list} 中。若未给定
      \VAR{list},收集至匿名列表并在终止后返回。
    }

    \IT{\Goo{\LKWD*{append}\XOR\LKWD*{appending}\XOR\LKWD*{nconc}\XOR\LKWD*{nconcing}}
      \Goo{\VAR{form}\XOR\LKWD*{it}} \Op{\LKWD*{into}
        \VAR{list}}}
    {
      以 \FU{append} 或 \FU{nconc} 的方式收集
      \VAR{form} 的值或 \LKWD{it} 至
      \VAR{list} 中,值应为列表。若未给定
      \VAR{list},收集至匿名列表并在终止后返回。
    }

    \IT{\Goo{\LKWD*{count}\XOR\LKWD*{counting}}
      \Goo{\VAR{form}\XOR\LKWD*{it}} \Op{\LKWD*{into}
        \VAR{n}} \Op{\VAR{type}}}
    {
      计算 \VAR{form} 的值或 \LKWD{it} 为
      \T\ 的次数。若未给定
      \VAR{n},计数至匿名变量并在终止后返回。
    }

    \IT{\Goo{\LKWD*{sum}\XOR\LKWD*{summing}}
      \Goo{\VAR{form}\XOR\LKWD*{it}} \Op{\LKWD*{into}
        \VAR{sum}} \Op{\VAR{type}}} 
    {
      计算 \VAR{form} 的主值或 \LKWD{it}
      的总和。若未给定
      \VAR{sum},累加至匿名变量并在终止后返回。
    }

    \IT{\Goo{\LKWD*{maximize}\XOR\LKWD*{maximizing}\XOR
        \LKWD*{minimize}\XOR
        \LKWD*{minimizing}} \Goo{\VAR{form}\XOR\LKWD*{it}} \Op{\LKWD*{into}
        \VAR{max-min}} \Op{\VAR{type}}} 
    {
      计算 \VAR{form} 的主值或 \LKWD{it}
      的最大/最小值。若未给定
      \VAR{max-min},使用匿名变量并在终止后返回。
    }

    \IT{\Goo{\LKWD*{initially}\XOR\LKWD*{finally}} \RP{\VAR{form}}}
    {
      在迭代开始前或结束后求值
      \VAR{form}s。
    }

    \IT{\LKWD*{repeat} \VAR{num}}
    {
      在 \VAR{num} 次迭代后终止 \MC{loop},\VAR{num}
      只求值一次。
    }

    \IT{\Goo{\LKWD*{while}\XOR\LKWD*{until}} \VAR{test}}
    {
      当 \VAR{test} 返回 \NIL\ 或
      \T 时终止迭代。
    }

    \IT{\Goo{\LKWD*{always}\XOR\LKWD*{never}} \VAR{test}}
    {
      一旦 \VAR{test} 为 \NIL\ 或 \T,终止
      \MC{loop} 并跳过任何 \LKWD{finally} 部分,返回
      \NIL。否则继续 \MC{loop},其默认返回值为
      \T。
    }

    \IT{\LKWD*{thereis} \VAR{test}}
    {
      当 \VAR{test} 为 \T\ 时,终止 \MC{loop}
      并跳过任何 \LKWD{finally} 部分,返回 \VAR{test}
      的值。否则继续 \MC{loop},其默认返回值为 \NIL。
    }

    \IT{(\MC*{loop-finish})}
    {
      立即终止 \MC{loop},执行 \LKWD{finally}
      语句,返回累积结果。
    }
  \end{LIST}
\end{LIST}




% LocalWords:  pt

%%% Local Variables: 
%%% mode: latex
%%% TeX-master: "clqr"
%%% End: 
