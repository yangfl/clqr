% Copyright (C) 2008, 2010, 2011, 2012, 2014 Bert Burgemeister
%
% Permission is granted to copy, distribute and/or modify this
% document under the terms of the GNU Free Documentation License,
% Version 1.2; with no Invariant Sections, no Front-Cover Texts and
% no Back-Cover Texts. For details see file COPYING.
%

%%%%%%%%%%%%%%%%%%%%%%%%%%%%%%%%%%%%%%%%%%%%%%%%%%
\section{外部环境}
%%%%%%%%%%%%%%%%%%%%%%%%%%%%%%%%%%%%%%%%%%%%%%%%%%
\begin{LIST}{1cm}

  \IT{\arrGOO{(\FU*{GET-INTERNAL-REAL-TIME})\\
      (\FU*{GET-INTERNAL-RUN-TIME})}{.}}
  {
    \retval{当前时间} 或 
    \retval{计算机时间},以时钟滴答表示。
  }
  
  \IT{\CNS*{INTERNAL-TIME-UNITS-PER-SECOND}}
  {
    每秒的时钟滴答数。
  }
  
  \IT{\arrGOO{%
      (\FU*{ENCODE-UNIVERSAL-TIME}
      \VAR{\hspace{.7ex}sec} \VAR{\hspace{.7ex}min} 
      \VAR{\hspace{.7ex}hour} \VAR{\hspace{.7ex}date}
      \VAR{\hspace{.7ex}month} \VAR{\hspace{.7ex}year\hspace{1ex}}
      \Op{\VAR{zone}\DF{curr}})\\ 
      (\FU*{GET-UNIVERSAL-TIME})}{.}}
  {
    \retval{自 1900-01-01, 00:00 起的秒数},忽略闰秒。
  }

  \IT{\arrGOO{(\FU*{DECODE-UNIVERSAL-TIME} \VAR{ universal-time } 
      \Op{\VAR{time-zone}\DF{current}})\\
    (\FU*{GET-DECODED-TIME})}{.}}
  {
    返回
    \retval{秒}、\retvalii{分}、\retvaliii{时}、\retvaln{4}{日}、\retvaln{5}{月}、\retvaln{6}{年}、\retvaln{7}{星期}、\retvaln{8}{夏令时-p}、\retvaln{9}{时区}。
  }

  \IT{\arrGOO{(\FU*{SHORT-SITE-NAME})\\
      (\FU*{LONG-SITE-NAME})}{.}}
  {
    表示计算机物理位置的 \retval{字符串}。
  }

  \IT{(\xorGOO{\FU{LISP-IMPLEMENTATION}\\
      \FU{SOFTWARE}\\
      \FU{MACHINE}}{\}}\kwd{-}%
    \xorGOO{\kwd{TYPE}\\
      \kwd{VERSION}}{\}})}
  {
    \index{LISP-IMPLEMENTATION-TYPE}%
    \index{LISP-IMPLEMENTATION-VERSION}%
    \index{SOFTWARE-TYPE}%
    \index{SOFTWARE-VERSION}%
    \index{MACHINE-TYPE}%
    \index{MACHINE-VERSION}%
    实现、操作系统、硬件的
    \retval{名称} 或 \retval{版本}。
  }

  \IT{(\FU*{MACHINE-INSTANCE})}
  {
    \retval{计算机名}。
  }

\end{LIST}


%%% Local Variables: 
%%% mode: latex
%%% TeX-master: "clqr"
%%% End: 
