% Copyright (C) 2008, 2009, 2010, 2011, 2012, 2014 Bert Burgemeister
%
% Permission is granted to copy, distribute and/or modify this
% document under the terms of the GNU Free Documentation License,
% Version 1.2; with no Invariant Sections, no Front-Cover Texts and
% no Back-Cover Texts. For details see file COPYING.
%

%%%%%%%%%%%%%%%%%%%%%%%%%%%%%%%%%%%%%%%%%%%%%%%%%%
\section{包及符号}
%%%%%%%%%%%%%%%%%%%%%%%%%%%%%%%%%%%%%%%%%%%%%%%%%%
Loop 宏提供了额外的符号处理方式,参见第 \pageref{section:Loop 宏} 页 \kwd{loop}。

%%%%%%%%%%%%%%%%%%%%%%%%%%%%%%%%%%%%%%%%%%%%%%%%%%
\subsection{谓词}
%%%%%%%%%%%%%%%%%%%%%%%%%%%%%%%%%%%%%%%%%%%%%%%%%%
\begin{LIST}{1cm}

  \IT{\arrGOO{(\FU*{SYMBOLP} \VAR{ foo})\\
      (\FU*{PACKAGEP} \VAR{ foo})\\
      (\FU*{KEYWORDP} \VAR{ foo})}{.}}
  {
    当 \VAR{foo} 为指定类型时为 \retval{\T}。
  }

\end{LIST}


%%%%%%%%%%%%%%%%%%%%%%%%%%%%%%%%%%%%%%%%%%%%%%%%%%
\subsection{包}
%%%%%%%%%%%%%%%%%%%%%%%%%%%%%%%%%%%%%%%%%%%%%%%%%%
\begin{LIST}{1cm}

  \IT{\kwd*{:}\VAR{bar}\XOR\kwd*{keyword}\kwd{:}\VAR{bar}}
  {
    关键字,结果为 \retval{:\VAR{bar}}。
  }

  \IT{\VAR{package}\kwd*{:}\VAR{symbol}}
  {
    \VAR{package} 的导出符号 \VAR{symbol}。
  }

  \IT{\VAR{package}\kwd*{::}\VAR{symbol}}
  {
    \VAR{package} 的(未)导出符号 \VAR{symbol}。
  }

  \IT{(\MC*{DEFPACKAGE}\VAR{ foo} 
    \orGOO{%
      \OPn{(\kwd{:nicknames }\OPn{\VAR{nick}})}\\
      (\kwd{:documentation }\VAR{string})\\
      \OPn{(\kwd{:intern }\OPn{\VAR{interned-symbol}})}\\
      \OPn{(\kwd{:use }\OPn{\VAR{used-package}})}\\
      \OPn{(\kwd{:import-from }\VAR{pkg } \OPn{\VAR{imported-symbol}})}\\
      \OPn{(\kwd{:shadowing-import-from}\VAR{ pkg}\OPn{\VAR{ shd-symbol}})}\\
      \OPn{(\kwd{:shadow }\OPn{\VAR{shd-symbol}})}\\
      \OPn{(\kwd{:export }\OPn{\VAR{exported-symbol}})}\\
      (\kwd{:size }\VAR{int})%
    }{\}})}
  {
    Create or modify \retval{package \VAR{foo}} with
    \VAR{interned-symbol}s, symbols from \VAR{used-package}s,
    \VAR{imported-symbol}s, and \VAR{shd-symbol}s. Add \VAR{shd-symbol}s
    to \VAR{foo}'s shadowing list.
  }

  \IT{(\FU*{MAKE-PACKAGE} \VAR{foo}
    \orGOO{\kwd{:nicknames }(\OPn{\VAR{nick}})\DF{\NIL}\\
      \kwd{:use }(\OPn{\VAR{used-package}})}{\}})}
  {
    创建 \retval{包 \VAR{foo}}。
  }

  \IT{(\FU*{RENAME-PACKAGE} \VAR{package} \VAR{new-name}
    \Op{\VAR{new-nicknames}\DF{\NIL}})}
  {
    重命名 \VAR{package}。返回 \retval{重命名的包}。
  }

  \IT{(\MC*{IN-PACKAGE }\NEV{\VAR{foo}})}
  {
    指定当前 \retval{包 \VAR{foo}}。
  }

  \IT{(\xorGOO{\FU*{USE-PACKAGE}\\
    \FU*{UNUSE-PACKAGE}}{\}} 
    \VAR{other-packages}
    \Op{\VAR{package}\DF{\V{\A package\A}}})}
  {
    Make exported symbols of \VAR{other-packages} available in
    \VAR{package}, or remove them from \VAR{package},
    respectively. Return \retval{\T}. 
  }

  \IT{\arrGOO{%
      (\FU*{PACKAGE-USE-LIST } \VAR{package})\\
      (\FU*{PACKAGE-USED-BY-LIST } \VAR{package})}{.}}
  {
  \retval{List of other packages} used by/using \VAR{package}.
  }

  \IT{(\FU*{DELETE-PACKAGE} \DES{\VAR{package}})}
  {
    删除 \VAR{package}。成功时返回 \retval{\T}。
  }

  \IT{\V{\A package\A}\DF{\kwd{common-lisp-user}}\qquad\qquad}
  {\index{*PACKAGE*@\A PACKAGE\A}
    当前包。
  }

  \IT{(\FU*{LIST-ALL-PACKAGES})\qquad\qquad\qquad}
  {
    \retval{已注册包列表}。
  }

  \IT{(\FU*{PACKAGE-NAME} \VAR{package})}
  {
    \retval{\VAR{package} 的名称}。
  }

  \IT{(\FU*{PACKAGE-NICKNAMES} \VAR{package})}
  {
    \VAR{package} 的 \retval{昵称}。
  }

  \IT{(\FU*{FIND-PACKAGE} \VAR{name})}
  {
    \VAR{name} \retval{包}(大小写敏感)。
  }

  \IT{(\FU*{FIND-ALL-SYMBOLS} \VAR{foo})}
  {
    \retval{List of symbols} \VAR{foo} from all
    registered packages.
  }

  \IT{(\xorGOO{\FU*{INTERN}\\
      \FU*{FIND-SYMBOL}}{\}}
    \VAR{foo} \Op{\VAR{package}\DF{\V{\A package\A}}})}
  {
    Intern or find, respectively, symbol \retval{\VAR{foo}} in
    \VAR{package}. Second return value is one of
    \retvalii{\kwd{:internal}}, \retvalii{\kwd{:external}}, or
    \retvalii{\kwd{:inherited}} (or \retvalii{\NIL} if \FU{intern}
    has created a fresh symbol).
  }

  \IT{(\FU*{UNINTERN} \VAR{symbol} \Op{\VAR{package}\DF{\V{\A package\A}}})}
  {
    Remove \VAR{symbol} from \VAR{package}, return \retval{\T} on success.
  }

  \IT{(\xorGOO{\FU*{IMPORT}\\
      \FU*{SHADOWING-IMPORT}}{\}} \VAR{symbols}
    \Op{\VAR{package}\DF{\V{\A package\A}}})}
  {
    Make \VAR{symbols} internal  to \VAR{package}. Return \retval{\T}. In
    case of a name conflict signal correctable \kwd{package-error} or shadow
    the old symbol, respectively.
  }

  \IT{(\FU*{SHADOW} \VAR{symbols}
    \Op{\VAR{package}\DF{\V{\A package\A}}})}
  {
    Make \VAR{symbols} of \VAR{package} shadow any otherwise
    accessible, equally named symbols from other
    packages. Return \retval{\T}.
  }

  \IT{(\FU*{PACKAGE-SHADOWING-SYMBOLS} \VAR{package})} 
  {
    \retval{List of symbols} of \VAR{package} that shadow any
    otherwise accessible, equally named symbols from other packages.
  }

  \IT{(\FU*{EXPORT} \VAR{symbols}
    \Op{\VAR{package}\DF{\V{\A package\A}}})}
  {
    Make \VAR{symbols} external to \VAR{package}. Return \retval{\T}.
  }

  \IT{(\FU*{UNEXPORT} \VAR{symbols} \Op{\VAR{package}\DF{\V{\A package\A}}})}
  {
    Revert \VAR{symbols} to internal status. Return \retval{\T}.
  }
  
  \IT{(\xorGOO{%
      \arrGOO{%
        \MC*{DO-SYMBOLS}\\
        \MC*{DO-EXTERNAL-SYMBOLS}}{\}}\text{ }
      (\NEV{\VAR{var}}\text{ } \OP{\VAR{package}\DF{\V{\A
            package\A}}\text{ } \Op{\VAR{result}\DF{\NIL}}})\\
      \MC*{DO-ALL-SYMBOLS } (\VAR{var }
      \Op{\VAR{result}\DF{\NIL}})}{\}}
    \OPn{(\kwd{declare} \OPn{\NEV{\VAR{decl}}})}
    \OPn{\orGOO{\NEV{\VAR{tag}}\\
      \VAR{form}}{\}}})}
  {
    Evaluate \SO{tagbody}-like body with \VAR{var} successively bound to every
    symbol from \VAR{package}, to every external symbol from
    \VAR{package}, or to every symbol from all registered packages, 
    respectively. Return \retval{values of \VAR{result}}.  Implicitly,
    the whole form is a \SO{block} named \NIL. 
  }

  \IT{(\MC*{WITH-PACKAGE-ITERATOR} (\VAR{foo} \VAR{packages}
    \Op{\kwd{:internal}\XOR\kwd{:external}\XOR\kwd{:inherited}})
    \OPn{(\kwd{declare} \OPn{\NEV{\VAR{decl}}})} \PROGN{\VAR{form}})}
  {
    Return \retval{values of \VAR{form}s}. In \VAR{form}s, successive
    invocations of  (\VAR{foo}) return: \T\ if a symbol is returned;
    a symbol from \VAR{packages}; accessibility
    (\kwd{:internal}, \kwd{:external}, or \kwd{:inherited}); and the
    package the symbol belongs to.
  }

  \IT{(\FU*{REQUIRE} \VAR{module} \Op{\VAR{paths}\DF{\NIL}})}
  {
    If not in \V{\A modules\A}, try \VAR{paths} to load
    \VAR{module} from. Signal \kwd{error} if
    unsuccessful. Deprecated. 
  }

  \IT{(\FU*{PROVIDE} \VAR{module})}
  {
    If not already there, add \VAR{module} to
    \V{\A modules\A}. Deprecated. 
  }

  \IT{\V{\A modules\A}}
  {\index{*MODULES*@\A MODULES\A}
    已加载模块名称列表。
  }

\end{LIST}


%%%%%%%%%%%%%%%%%%%%%%%%%%%%%%%%%%%%%%%%%%%%%%%%%%
\subsection{符号}
%%%%%%%%%%%%%%%%%%%%%%%%%%%%%%%%%%%%%%%%%%%%%%%%%%
\kwd*{symbol} 具有属性 \VAR{name}、主 \kwd{package}、属性
表,及可选的值(全局常量或变量
\VAR{name})、函数(\kwd{function}、macro、special operator \VAR{name})。

\begin{LIST}{1cm}

  \IT{(\FU*{MAKE-SYMBOL} \VAR{name})}
  {
    生成新的未进入 \retval{符号 \VAR{name}}。
  }

  \IT{(\FU*{GENSYM} \Op{\VAR{s}\DF{\LIT{G}}})}
  {
    Return fresh, uninterned symbol \retval{\kwd{\#:}\VAR{s}\VAR{n}}
    with \VAR{n} from \V{\A gensym-counter\A}. Increment \V{\A
      gensym-counter\A}.%
    \index{*GENSYM-COUNTER*@\A GENSYM-COUNTER\A}
  }

  \IT{(\FU*{GENTEMP} \OP{\VAR{prefix}\DF{\LIT{T}}
      \Op{\VAR{package}\DF{\V{\A package\A}}}})}
  {
    Intern fresh \retval{symbol} in \retval{package}. Deprecated.
  }

  \IT{(\FU*{COPY-SYMBOL} \VAR{symbol} \Op{\VAR{props}\DF{\NIL}})}
  {
    Return uninterned \retval{copy of \VAR{symbol}}. If \VAR{props} is
    \T, give copy the same value, function and property list.
  }

  \IT{\arrGOO{%
      (\FU*{SYMBOL-NAME} \VAR{ symbol})\\
      (\FU*{SYMBOL-PACKAGE} \VAR{ symbol})\\
      (\FU*{SYMBOL-PLIST} \VAR{ symbol})\\
      (\FU*{SYMBOL-VALUE} \VAR{ symbol})\\
      (\FU*{SYMBOL-FUNCTION} \VAR{ symbol})}{.}}
  {
  \VAR{symbol} 的
  \retval{名称}、\retval{包}、\retval{属性表}、
  \retval{值}、\retval{函数}。可 \kwd{setf}。
  }

  \IT{(\xorGOO{%
      \GFU*{DOCUMENTATION}\\
      (\kwd{SETF } \GFU{DOCUMENTATION})\VAR{ new-doc}}{\}}
    \VAR{foo}
    \xorGOO{%
      \kwd{'variable}\XOR
      \kwd{'function}\\
      \kwd{'compiler-macro}\\
      \kwd{'method-combination}\\
      \kwd{'structure}\XOR
      \kwd{'type}\XOR
      \kwd{'setf}\XOR
      \T}{\}})}
  {\index{VARIABLE}\index{FUNCTION}\index{COMPILER-MACRO}\index{METHOD-COMBINATION}\index{STRUCTURE}\index{TYPE}\index{SETF}
    获取/设置给定类型
    \VAR{foo} 的 \retval{文档字符串}。
  }

  \IT{\CNS*{T}}
  {
    Truth; the supertype of every type including \kwd{t}; the
    superclass of every class except \kwd{t}; \V{\A terminal-io\A}.
  }

  \IT{\CNS*{NIL}\XOR{\CNS*{()}}}
  {
    Falsity; the empty list; the empty type, subtype of every type;
    \V{\A standard-input\A}; \V{\A standard-output\A}; the global
    environment. 
  }

\end{LIST}


%%%%%%%%%%%%%%%%%%%%%%%%%%%%%%%%%%%%%%%%%%%%%%%%%%
\subsection{标准包}
%%%%%%%%%%%%%%%%%%%%%%%%%%%%%%%%%%%%%%%%%%%%%%%%%%
\begin{LIST}{1cm}

  \IT{\kwd*{COMMON-LISP}\XOR\kwd*{CL}}
  {
    Exports the defined names of Common Lisp except for those in the
    \kwd{keyword} package.
  }

  \IT{\kwd*{COMMON-LISP-USER}\XOR\kwd*{CL-USER}}
  {
    启动后的当前包,使用包 \kwd{common-lisp}。
  }

  \IT{\kwd*{KEYWORD}}
  {
    含有定义为类型 \kwd{KEYWORD} 的符号。
  }

\end{LIST}

%%% Local Variables: 
%%% mode: latex
%%% TeX-master: "clqr"
%%% End: 
