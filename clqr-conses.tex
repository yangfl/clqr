% Copyright (C) 2008, 2009, 2010, 2011, 2012, 2014 Bert Burgemeister
%
% Permission is granted to copy, distribute and/or modify this
% document under the terms of the GNU Free Documentation License,
% Version 1.2; with no Invariant Sections, no Front-Cover Texts and
% no Back-Cover Texts. For details see file COPYING.
%

%%%%%%%%%%%%%%%%%%%%%%%%%%%%%%%%%%%%%%%%%%%%%%%%%%
\section{点对} 
%%%%%%%%%%%%%%%%%%%%%%%%%%%%%%%%%%%%%%%%%%%%%%%%%%

%%%%%%%%%%%%%%%%%%%%%%%%%%%%%%%%%%%%%%%%%%%%%%%%%%
\subsection{谓词} 
%%%%%%%%%%%%%%%%%%%%%%%%%%%%%%%%%%%%%%%%%%%%%%%%%%
\begin{LIST}{1cm}

  \IT{\arrGOO{(\FU*{CONSP} \VAR{ foo})\\
      (\FU*{LISTP} \VAR{ foo})}{.}}
  {
    当 \VAR{foo} 为指定类型时为 \retval{\T}。
    }

  \IT{\arrGOO{(\FU*{ENDP} \VAR{ list})\\
      (\FU*{NULL} \VAR{ foo})}{.}\qquad\qquad}
  {
  当 \VAR{list}/\VAR{foo} 为 \NIL\ 时为 \retval{\T}。
  }

  \IT{(\FU*{ATOM} \VAR{foo})\qquad}
  {当 \VAR{list}/\VAR{foo} 不为 \kwd{cons} 时返回
  \retval{\T}。
  }

  \IT{(\FU*{TAILP} \VAR{foo} \VAR{list})}
  {
  当 \VAR{foo} 为 \VAR{list} 的尾部时返回 \retval{\T}。
  }

  \IT{(\FU*{MEMBER} \VAR{foo} \VAR{list}
    \orGOO{\xorGOO{%
        \kwd{:test} \VAR{ function}\DF{\kwd{\#'eql}}\\
        \kwd{:test-not} \VAR{ function}}{.}\\
      \kwd{:key} \VAR{ function}}{\}})}
  {
  返回第一个元素匹配 \VAR{foo} 的
  \retval{\VAR{list} 尾部}。如果不存在这样的元素,返回
  \retval{\NIL}。
  }

  \IT{(\xorGOO{\FU*{MEMBER-IF}\\
      \FU*{MEMBER-IF-NOT}}{\}}
    \VAR{test} \VAR{list}
    \Op{\kwd{:key} \VAR{function}})}
  {
  返回第一个元素满足 \VAR{test} 的
  \retval{\VAR{list} 尾部}。如果不存在这样的元素,返回
  \retval{\NIL}。
  }

  \IT{(\FU*{SUBSETP} \VAR{list-a} \VAR{list-b} 
    \orGOO{\xorGOO{%
        \kwd{:test} \VAR{ function}\DF{\kwd{\#'eql}}\\
        \kwd{:test-not} \VAR{ function}}{.}\\
      \kwd{:key} \VAR{ function}}{\}})}
  {
  当 \VAR{list-a} 为 \VAR{list-b} 的子集时返回
  \retval{\T}。
  }

\end{LIST}


%%%%%%%%%%%%%%%%%%%%%%%%%%%%%%%%%%%%%%%%%%%%%%%%%%
\subsection{列表} 
%%%%%%%%%%%%%%%%%%%%%%%%%%%%%%%%%%%%%%%%%%%%%%%%%%
\begin{LIST}{1cm}

  \IT{(\FU*{CONS} \VAR{foo} \VAR{bar})}
  {
  返回新点对 \retval{(\VAR{foo} \kwd {.} \VAR{bar})}。
  }

  \IT{(\FU*{LIST} \OPn{\VAR{foo}})\qquad\qquad}
  {返回 \retval{\VAR{foo}s 的列表}。
    }

  \IT{(\FU*{LIST\A} \RP{\VAR{foo}})}
  {返回 \retval{\VAR{foo}s 的列表},其中最后一个的
  \VAR{foo} 是最后一个点对的 cdr。如果只有一个
  \VAR{foo},返回 \retval{\VAR{foo}}。
  }

  \IT{(\FU*{MAKE-LIST} \VAR{num} \Op{\kwd{:initial-element}
      \VAR{foo}\DF{\NIL}})}
  {
    有 \VAR{num} 个 \VAR{foo} 元素的新 \retval{list}。
  }

  \IT{(\FU*{LIST-LENGTH} \VAR{list})}
  {\VAR{list} 的 \retval{长度},循环(circular)
  \VAR{list} 为 \retval{\NIL}。
  }

  \IT{(\FU*{CAR} \VAR{list})\qquad\qquad}
  {
  \retval{\VAR{list} 的 car},如果 \VAR{list} 为 \retval{\NIL},则为
  \NIL。可 \kwd{SETF}。
  }

  \IT{\arrGOO{(\FU*{CDR} \VAR{ list})\qquad\\
      (\FU*{REST} \VAR{ list})}{.}}
  {
  \retval{\VAR{list} 的 cdr},如果 \VAR{list} 为 \retval{\NIL},则为
  \NIL。可 \kwd{SETF}。
  }

  \IT{(\FU*{NTHCDR} \VAR{n list})}
  {在调用 \FU{cdr} \VAR{n} 次后返回 \retval{\VAR{list} 的尾部}。
    }

  \IT{(\Goo{\FU*{FIRST}\XOR\FU*{SECOND}\XOR\FU*{THIRD}\XOR\FU*{FOURTH}\XOR\FU*{FIFTH}\XOR\FU*{SIXTH}\XOR\dots\XOR\FU*{NINTH}\XOR\FU*{TENTH}}
    \VAR{list})}
  {
  \index{SEVENTH}%
  \index{EIGHTH}%
  若存在,返回 \retval{\VAR{list} 的第 n 个元素},否则返回
  \retval{\NIL}。可 \kwd{SETF}。
  }

  \IT{(\FU*{NTH} \VAR{n list})}
  {
    \VAR{list} 的零索引 \retval{第\ \VAR{n} 个元素}。可 \kwd{setf}。
  }

  \IT{(\FU{C}\VAR{X}\kwd{R} \VAR{list})}
  {
  \index{CAAR}%
  \index{CADR}%
  \index{CDAR}%
  \index{CDDR}%
  \VAR{X} 可以是一至四个
  \kwd{a} 和 \kwd{d},分别表示\FU{CAR} 和 \FU{CDR}。如
  (\FU{CADR} \VAR{bar}) 等价于 (\FU{CAR} (\FU{CDR}
  \VAR{bar}))。可
  \kwd{SETF}。
  }

  \IT{(\FU*{LAST} \VAR{list} \Op{\VAR{num}\DF{\LIT{1}}})}
  {
  返回 \VAR{list} \retval{最后\ \VAR{num}
    个点对} 的列表。
  }

  \IT{(\xorGOO{\FU*{BUTLAST }  \VAR{list}\\
      \FU*{NBUTLAST }  \DES{\VAR{list}}}{\}}
    \Op{\VAR{num}\DF{\LIT{1}}})}
  {
    \retval{\VAR{list}},排除最后 \VAR{num}
    个点对。
  }

  \IT{(\xorGOO{\FU*{RPLACA}\\
    \FU*{RPLACD}}{\}} \DES{\VAR{cons}} \VAR{object})}
  {
  用 \VAR{object} 替换 \retval{\VAR{cons}} 的 car 或 cdr。
  }

  \IT{(\FU*{LDIFF} \VAR{list} \VAR{foo})}
  {
  如果 \VAR{foo} 是 \VAR{list} 的尾部,返回 \retval{\VAR{list}
    的前一部分}。否则,返回 \retval{\VAR{list}}。
  }

  \IT{(\FU*{ADJOIN} \VAR{foo} \VAR{list} 
    \orGOO{\xorGOO{%
        \kwd{:test} \VAR{ function}\DF{\kwd{\#'eql}}\\
        \kwd{:test-not} \VAR{ function}}{.}\\
      \kwd{:key} \VAR{ function}}{\}})}
  {如果 \VAR{foo} 已经在 \VAR{list} 中,返回
  \retval{\VAR{list}}。如果不存在,返回 \retval{(\FU{CONS}
    \VAR{foo} \VAR{list})}。
  }

  \IT{(\MC*{POP} \DES{\VAR{place}})}
  {
    将 \VAR{place} 设为 (\FU{CDR} \VAR{place}),返回
    \retval{(\FU{CAR} \VAR{place})}。
  }

  \IT{(\MC*{PUSH} \VAR{foo} \DES{\VAR{place}})}
  {
    将 \VAR{place} 设为 \retval{(\FU{cons} \VAR{foo} \VAR{place})}。
  }

  \IT{(\MC*{PUSHNEW} \VAR{foo} \DES{\VAR{place}}  
    \orGOO{\xorGOO{%
        \kwd{:test} \VAR{ function}\DF{\kwd{\#'eql}}\\
        \kwd{:test-not} \VAR{ function}}{.}\\
      \kwd{:key} \VAR{ function}}{\}})}
  {
    将 \VAR{place} 设为 \retval{(\FU{adjoin} \VAR{foo} \VAR{place})}。
  }

  \IT{\arrGOO{(\FU*{APPEND } \Op{\OPn{\VAR{proper-list}} 
        \VAR{ foo}\DF{\NIL}})\\
      (\FU*{NCONC } \Op{\OPn{\DES{\VAR{non-circular-list}}} 
        \VAR{ foo}\DF{\NIL}})}{.}}
  {
    返回 \retval{串接列表},只有一个参数时,返回
    \retval{\VAR{foo}}。\VAR{foo} 可以是任意类型。
  }

  \IT{\arrGOO{(\FU*{REVAPPEND} \VAR{ list} \VAR{ foo})\\
      (\FU*{NRECONC }
      \DES{\VAR{list}} \VAR{ foo})}{.}}
  {
    在反转 \VAR{list} 的顺序后返回
    \retval{串接的列表}。
  }

  \IT{(\xorGOO{\FU*{MAPCAR}\\
      \FU*{MAPLIST}}{\}} \VAR{function} \RP{\VAR{list}})}
  {
    用 \VAR{list}s 的相应 car 或 cdr 依次调用
    \VAR{function},返回
    \retval{返回值的列表}。
  }

  \IT{(\xorGOO{\FU*{MAPCAN}\\
      \FU*{MAPCON}}{\}} \VAR{function} \RP{\VAR{\DES{list}}})}
  % Example of list argument being modified:
  %   (let ((list '((1 2) (3 4))))
  %       (mapcan #'identity list)
  %       list)
  {
    用 \VAR{list}s 的相应 car 或 cdr 依次调用
    \VAR{function},返回
    \retval{串接的返回值列表}。\VAR{function}
    应返回列表。
  }

  \IT{(\xorGOO{\FU*{MAPC}\\
      \FU*{MAPL}}{\}} \VAR{function} \RP{\VAR{list}})}
  {
    用 \VAR{list}s 的相应 car 或 cdr 依次调用
    \VAR{function},返回
    \retval{第一个 \VAR{list}}。\VAR{function}
    应有一定的副作用。
  }

  \IT{(\FU*{COPY-LIST} \VAR{list})}
  {
    返回共享元素的 \VAR{list} \retval{副本}。
  }

\end{LIST}


%%%%%%%%%%%%%%%%%%%%%%%%%%%%%%%%%%%%%%%%%%%%%%%%%%
\subsection{关联表} 
%%%%%%%%%%%%%%%%%%%%%%%%%%%%%%%%%%%%%%%%%%%%%%%%%%
\label{section:关联表}
\begin{LIST}{1cm}

  \IT{(\FU*{PAIRLIS} \VAR{keys} \VAR{values} \Op{\VAR{alist}\DF{\NIL}})}
  {
    向 \retval{\VAR{alist}} 前加上从列表
    \VAR{keys} 和 \VAR{values} 生成的关联表。
  }

  \IT{(\FU*{ACONS} \VAR{key} \VAR{value} \VAR{alist})}
  {
    返回加上了 (\VAR{key} \kwd{.} \VAR{value}) 对的
    \retval{\VAR{alist}}。
  }

  \IT{\arrGOO{(\xorGOO{\FU*{ASSOC}\\
        \FU*{RASSOC}}{\}}
      \VAR{ foo} \VAR{ alist }
      \orGOO{\xorGOO{%
          \kwd{:test} \VAR{ test}\DF{\kwd{\#'eql}}\\
          \kwd{:test-not} \VAR{ test}}{.}\\
        \kwd{:key} \VAR{ function}
      }{\}})\\
    (\xorGOO{\FU*{ASSOC-IF}\Op{\kwd{-NOT}}\\
      \FU*{RASSOC-IF}\Op{\kwd{-NOT}}}{\}} \VAR{ test} \VAR{ alist }
      \Op{\kwd{:key} \VAR{ function}})}{.}}
  {\index{ASSOC-IF-NOT}\index{RASSOC-IF-NOT}%
    第一个其 car 或 cdr 满足
    \VAR{test} 的 \retval{cons}。
  }

  \IT{(\FU*{COPY-ALIST} \VAR{alist})}
  {
    返回 \VAR{alist} 的 \retval{副本}。
  }

\end{LIST}


%%%%%%%%%%%%%%%%%%%%%%%%%%%%%%%%%%%%%%%%%%%%%%%%%%
\subsection{树} 
%%%%%%%%%%%%%%%%%%%%%%%%%%%%%%%%%%%%%%%%%%%%%%%%%%
\begin{LIST}{1cm}

  \IT{(\FU*{TREE-EQUAL} \VAR{foo} \VAR{bar} 
    \xorGOO{\kwd{:test} \VAR{ test}\DF{\kwd{\#'eql}}\\
      \kwd{:test-not} \VAR{ test}}{\}})}
  {
    当树 \VAR{foo} 和 \VAR{bar} 有满足
    \VAR{test} 的相同形状和叶子时返回 \retval{\T}。
  }

  \IT{(\xorGOO{\FU*{SUBST} \VAR{ new} \VAR{ old } \VAR{tree}\\
      \FU*{NSUBST} \VAR{ new} \VAR{ old } \DES{\VAR{tree}}}{\}}
    \orGOO{\xorGOO{%
        \kwd{:test} \VAR{ function}\DF{\kwd{\#'eql}}\\
        \kwd{:test-not} \VAR{ function}}{.}\\
      \kwd{:key} \VAR{ function}%
    }{\}})}
  {
    生成 \retval{\VAR{tree} 的副本},其中用
    \VAR{new} 替换了匹配 \VAR{old} 的子树或叶子。
  }

  \IT{(\xorGOO{\FU{SUBST-IF\Op{-NOT}} \VAR{ new} \VAR{ test } \VAR{tree}\\
      \FU{NSUBST-IF\Op{-NOT}} \VAR{ new} \VAR{ test } \DES{\VAR{tree}}}{\}} 
    \Op{\kwd{:key} \VAR{function}})}
  {
  \index{SUBST-IF}%
  \index{SUBST-IF-NOT}%
  \index{NSUBST-IF}%
  \index{NSUBST-IF-NOT}%
  生成 \retval{\VAR{tree} 的副本},其中用
  \VAR{new} 替换了满足 \VAR{test} 的子树或叶子。
  }

  \IT{(\xorGOO{\FU*{SUBLIS} \VAR{ association-list } \VAR{tree} \\
      \FU*{NSUBLIS} \VAR{ association-list } \DES{\VAR{tree}} }{\}}
    \orGOO{\xorGOO{%
        \kwd{:test} \VAR{ function}\DF{\kwd{\#'eql}}\\
        \kwd{:test-not} \VAR{ function}}{.}\\
      \kwd{:key} \VAR{ function}%
    }{\}})}
  {
  生成 \retval{\VAR{tree} 的副本},其中用
  \VAR{association-list} 中的值替换了匹配其键的子树或叶子。
  }

  \IT{(\FU*{COPY-TREE} \VAR{tree})} 
  {
    返回具有相同形状和叶子的 \retval{\VAR{tree} 副本}。
  }

\end{LIST}


%%%%%%%%%%%%%%%%%%%%%%%%%%%%%%%%%%%%%%%%%%%%%%%%%%
\subsection{集合} 
%%%%%%%%%%%%%%%%%%%%%%%%%%%%%%%%%%%%%%%%%%%%%%%%%%
\begin{LIST}{1cm}

  \IT{(\xorGOO{%
      \arrGOO{%
        \FU*{INTERSECTION}\\
        \FU*{SET-DIFFERENCE}\\
        \FU*{UNION}\\
        \FU*{SET-EXCLUSIVE-OR}%
      }{\}} \VAR{ a} \VAR{ b}\\
      \arrGOO{%
        \FU*{NINTERSECTION}\\
        \FU*{NSET-DIFFERENCE}%
      }{\}} \text{ }\DES{\VAR{a}}\text{ } \VAR{b}\\
      \arrGOO{%
        \FU*{NUNION}\\
        \FU*{NSET-EXCLUSIVE-OR}%
      }{\}} \text{ }\DES{\VAR{a}}\text{ } \DES{\VAR{b}}\text{ }
    }{\}}
    \orGOO{\xorGOO{%
        \kwd{:test} \VAR{ function}\DF{\kwd{\#'eql}}\\
        \kwd{:test-not} \VAR{ function}}{.}\\
      \kwd{:key} \VAR{ function}}{\}})}
  {
  返回 \VAR{a} 和 \VAR{b} 的
  \retval{$a\cap b$}、\retval{$a\setminus b$}、\retval{$a\cup b$} 或 \retval{$a\,\triangle\, b$}。
  }

\end{LIST}


%%% Local Variables: 
%%% mode: latex
%%% TeX-master: "clqr"
%%% End: 
