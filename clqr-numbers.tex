% Copyright (C) 2008, 2009, 2010, 2011, 2012, 2014 Bert Burgemeister
%
% Permission is granted to copy, distribute and/or modify this
% document under the terms of the GNU Free Documentation License,
% Version 1.2; with no Invariant Sections, no Front-Cover Texts and
% no Back-Cover Texts. For details see file COPYING.
%

%%%%%%%%%%%%%%%%%%%%%%%%%%%%%%%%%%%%%%%%%%%%%%%%%%
\section{数} 
%%%%%%%%%%%%%%%%%%%%%%%%%%%%%%%%%%%%%%%%%%%%%%%%%%

%%%%%%%%%%%%%%%%%%%%%%%%%%%%%%%%%%%%%%%%%%%%%%%%%%
\subsection{谓词} 
%%%%%%%%%%%%%%%%%%%%%%%%%%%%%%%%%%%%%%%%%%%%%%%%%%

\begin{LIST}{1cm}

  \IT{\arrGOO{(\FU*{=}\RP{\VAR{
          number}})\\
      (\FU*{/=}\RP{\VAR{ number}})}{.}}
     {
       当所有 \VAR{number}s
       相等(或不等时)为 \retval{\T}。
     }

  \IT{\arrGOO{(\FU{\boldmath$>$}\RP{\VAR{
          number}})\\(\FU{\boldmath$>$=}\RP{\VAR{
          number}})\\(\FU{\boldmath$<$}\RP{\VAR{
          number}})\\(\FU{\boldmath$<$=}\RP{\VAR{ number}})}{.}}
  {
  \index{>@$>$}%
  \index{>=@$>$=}%
  \index{<@$<$}%
  \index{<=@$<$=}%
  当 \VAR{number}s
  严格单调减、非严格单调减、严格单调增、非严格单调增时为
  \retval{\T}。
  }

  \IT{\arrGOO{(\FU*{MINUSP} \VAR{ a})\\
      (\FU*{ZEROP} \VAR{ a})\\
      (\FU*{PLUSP}
      \VAR{ a})}{.}}
  {
  当 $a < 0$、$a = 0$ 或 $a > 0$ 时为 \retval{\T}。
  }

  \IT{\arrGOO{(\FU*{EVENP } \VAR{int})\\
      (\FU*{ODDP } \VAR{int})}{.}}
  {
    当 \VAR{int} 为偶数或奇数时为 \retval{\T}。
  }

  \IT{\arrGOO{(\FU*{NUMBERP} \VAR{ foo})\\
      (\FU*{REALP} \VAR{ foo})\\
      (\FU*{RATIONALP} \VAR{ foo})\\
      (\FU*{FLOATP} \VAR{ foo})\\
      (\FU*{INTEGERP} \VAR{ foo})\\
      (\FU*{COMPLEXP} \VAR{ foo})\\
      (\FU*{RANDOM-STATE-P} \VAR{ foo})
      }{.}}
  {
  当 \VAR{foo}
  为指定类型时为 \retval{\T}。
  }

\end{LIST}

%%%%%%%%%%%%%%%%%%%%%%%%%%%%%%%%%%%%%%%%%%%%%%%%%%
\subsection{数值函数} 
%%%%%%%%%%%%%%%%%%%%%%%%%%%%%%%%%%%%%%%%%%%%%%%%%%

\begin{LIST}{1cm}

  \IT{\arrGOO{(\FU*{+} \OPn{\VAR{ a}\DF{\LIT{0}}})\\
      (\FU{\A} \OPn{\VAR{ a}\DF{\LIT{1}}})}{.}}
  {\index{*@\A}
  返回 \retval{$\sum{a}$} 或 \retval{$\prod{a}$}。
  }

  \IT{\arrGOO{(\FU*{--} \VAR{ a}\OPn{\VAR{ b}})\\
      (\FU*{/} \VAR{ a}
      \OPn{\VAR{ b}})}{.}}
  {
  返回 \retval{$a-\sum{b}$} 或 \retval{$a/\prod{b}$}。没有参数
  \VAR{b}s 时,返回 \retval{$-a$} 或 \retval{$1/a$}。
  }

  \IT{\arrGOO{(\FU*{1+} \VAR{ a})\\(\FU*{1--} \VAR{ a})}{.}}
  {返回 \retval{$a+1$}
  或 \retval{$a-1$}。
  }

  \IT{(\xorGOO{\MC*{INCF}\\
      \MC*{DECF}}{\}} \DES{\VAR{place}}
    \Op{\VAR{delta}\DF{\LIT{1}}})}
  {
  将 \VAR{place} 的值增加或减少 \VAR{delta}。返回 \retval{新值}。
  }

  \IT{\arrGOO{%
      (\FU*{EXP } \VAR{p})\\
      (\FU*{EXPT } \VAR{b } \VAR{p})}{.}\qquad\qquad}
     {
       返回 \retval{$\mbox{e}^p$} 或 \retval{$b^p$}。
     }

  \IT{(\FU*{LOG} \VAR{a} \Op{\VAR{b}\DF{e}})}
  {
    返回 \retval{$\log_b a$}。不带参数
    \VAR{b} 等同于 \retval{$\ln a$}。
  }

  \IT{\arrGOO{(\FU*{SQRT} \VAR{ n})\\
      (\FU*{ISQRT} \VAR{ n})}{.}\qquad\qquad}
  {
    复数/自然数 \retval{$\sqrt{n}$}。
  }

  \IT{\arrGOO{(\FU*{LCM} \OPn{\VAR{ integer}}\DF{\LIT{1}})\\
      (\FU*{GCD} \OPn{\VAR{ integer}})}{.}}
  {
  \VAR{integer}s 的 \retval{最小公倍数} 或
  \retval{最大公约数}。(\kwd{gcd})
  返回 \retval{0}。
  }

  \IT{\CNS*{PI}\qquad\qquad}
  {
  $\pi$ 的 \kwd{long-float} 近似,鲁道夫结果。
  }

  \IT{\arrGOO{(\FU*{SIN} \VAR{ a})\\
      (\FU*{COS} \VAR{ a})\\
      (\FU*{TAN} \VAR{ a})}{.}}
  {
    \retval{$\sin a$}、\retval{$\cos
    a$}、\retval{$\tan a$}。(其中 \VAR{a} 为弧度制)
  }

  \IT{\arrGOO{(\FU*{ASIN} \VAR{ a})\\
      (\FU*{ACOS} \VAR{ a})}{.}}
  {
  \retval{$\arcsin a$}、\retval{$\arccos
    a$},弧度制。
  }

  \IT{(\FU*{ATAN} \VAR{a} \Op{\VAR{b}\DF{\LIT{1}}})}
  {
  \retval{$\arctan \frac{a}{b}$},弧度制。
  }

  \IT{\arrGOO{(\FU*{SINH} \VAR{ a})\\(\FU*{COSH} \VAR{ a})\\(\FU*{TANH}
      \VAR{ a})}{.}}
  {
  \retval{$\sinh a$}、\retval{$\cosh
    a$}、\retval{$\tanh a$}。
  }

  \IT{\arrGOO{(\FU*{ASINH} \VAR{ a})\\
      (\FU*{ACOSH} \VAR{ a})
      \\(\FU*{ATANH} \VAR{ a})}{.}}
  {
  \retval{$\operatorname{asinh} a$}、\retval{$\operatorname{acosh}
    a$}、\retval{$\operatorname{atanh} a$}。
  }

  \IT{(\FU*{CIS} \VAR{a})\qquad\qquad}
  {
  返回
  \retval{$\operatorname{e}^{\operatorname{i} a}$} $=$ \retval{$\cos a +
    \operatorname{i}\sin a$}.
  }

  \IT{(\FU*{CONJUGATE} \VAR{a})}
  {
    返回复数 \retval{\VAR{a} 的共轭}。
    }

  \IT{\arrGOO{(\FU*{MAX } \RP{\VAR{num}})\\
      (\FU*{MIN } \RP{\VAR{num}})}{.}}
  {
  \VAR{num}s 的 \retval{最大} 或 \retval{最小} 值。
  }

  \IT{(\xorGOO{%
      \Goo{\FU*{ROUND}\XOR\FU*{FROUND}}\\
      \Goo{\FU*{FLOOR}\XOR\FU*{FFLOOR}}\\
      \Goo{\FU*{CEILING}\XOR\FU*{FCEILING}}\\
      \Goo{\FU*{TRUNCATE}\XOR\FU*{FTRUNCATE}}}{\}}
    \VAR{n} \Op{\VAR{d}\DF{\LIT{1}}})}
  {
  返回 \kwd{integer} 或 \kwd{float},及其
  \retvalii{余数}。取整方法分别为四舍五入及向
  $-\infty$、$+\infty$、$0$ 取整。
  }

  \IT{(\xorGOO{\FU*{MOD}\\
      \FU*{REM}}{\}} \VAR{n} \VAR{d})}
  {与 \FU{floor} 或 \FU{truncate}
  相同,但只返回 \retval{余数}。
  }

  \IT{(\FU*{RANDOM} \VAR{limit} \Op{\DES{\VAR{state}}\DF{\V{\A random-state\A}}})} 
  {
    返回小于 \VAR{limit} 的相同类型非负
    \retval{随机数}。
  }

  \IT{(\FU*{MAKE-RANDOM-STATE} \OP{\Goo{\VAR{state}\XOR\NIL\XOR\T}\DF{\NIL}})}
  {
    当前随机状态对象的
    \retval{副本};当前随机状态;或随机初始化新的
    \retval{随机状态}。
  }

  \IT{\V{\A random-state\A}\qquad\qquad\qquad}
  {\index{*RANDOM-STATE*@\A RANDOM-STATE\A}
    当前随机状态。
  }

  \IT{(\FU*{FLOAT-SIGN} \VAR{num-a} \Op{\VAR{num-b}\DF{\LIT{1}}})}
  {
  \retval{\VAR{num-b}},符号与 \VAR{num-a} 相同。
  }

  \IT{(\FU*{SIGNUM} \VAR{n})}
  {绝对值为 1、符号或角度与
  \VAR{n} 相同的 \retval{数}。
  }

  \IT{\arrGOO{(\FU*{NUMERATOR} \VAR{ rational})\\
      (\FU*{DENOMINATOR} \VAR{ rational})}{.}}
  {
  \VAR{rational} 既约形式的
  \retval{分子} 或 \retval{分母}。
  }

  \IT{\arrGOO{(\FU*{REALPART} \VAR{ number})\\
      (\FU*{IMAGPART} \VAR{ number})}{.}}
  {
  \VAR{number} 的 \retval{实部} 或 \retval{虚部}。
  }

  \IT{(\FU*{COMPLEX} \VAR{real} \Op{\VAR{imag}\DF{\LIT{0}}})}
  {
    生成 \retval{复数}。
  }

  \IT{(\FU*{PHASE} \VAR{num})}
  {
    \VAR{num} 在极坐标中的 \retval{角度}。
  }

  \IT{(\FU*{ABS} \VAR{n})\qquad\qquad}
  {
    返回 \retval{$|n|$}。
  }

  \IT{\arrGOO{(\FU*{RATIONAL} \VAR{ real})\\
      (\FU*{RATIONALIZE} \VAR{ real})}{.}}
  {
  将 \VAR{real} 转换为 \retval{分数}。分别为 \VAR{real} 的精确及近似值。
  }

  \IT{(\FU*{FLOAT} \VAR{real}
    \Op{\VAR{prototype}\DF{\LIT{0.0F0}}})}
  {
  将 \VAR{real} 转换为类型为 \VAR{prototype} 的 \retval{浮点数}。
  }

\end{LIST}


%%%%%%%%%%%%%%%%%%%%%%%%%%%%%%%%%%%%%%%%%%%%%%%%%%
\subsection{逻辑函数} 
%%%%%%%%%%%%%%%%%%%%%%%%%%%%%%%%%%%%%%%%%%%%%%%%%%
\label{section:逻辑函数}
负整数以二进制补码表示。

\begin{LIST}{1cm}

  \IT{(\FU*{BOOLE} \VAR{operation} \VAR{int-a} \VAR{int-b})}
  {
  返回逻辑运算 \VAR{operation} 所得的
  \retval{值}。\VAR{operation}s
  可以是
  }
  
  \begin{LIST}{.5cm}
    \IT{\CNS*{BOOLE-1}\qquad\qquad} {\retval{\VAR{int-a}}。}
    \IT{\CNS*{BOOLE-2}\qquad\qquad} {\retval{\VAR{int-b}}。}
    \IT{\CNS*{BOOLE-C1}\qquad\qquad} {\retval{$\lnot\text{\VAR{int-a}}$}。}
    \IT{\CNS*{BOOLE-C2}\qquad\qquad} {\retval{$\lnot\text{\VAR{int-b}}$}。}
    \IT{\CNS*{BOOLE-SET}\qquad\qquad} {\retval{所有位全 1}。}
    \IT{\CNS*{BOOLE-CLR}\qquad\qquad} {\retval{所有位全 0}。}
    \IT{\CNS*{BOOLE-EQV}\qquad\qquad} {\retval{$\text{\VAR{int-a}} \equiv \text{\VAR{int-b}}$}。}
    \IT{\CNS*{BOOLE-AND}\qquad\qquad} {\retval{$\text{\VAR{int-a}}\land\text{\VAR{int-b}}$}。}
    \IT{\CNS*{BOOLE-ANDC1}} {\retval{$\lnot \text{\VAR{int-a}} \land \text{\VAR{int-b}}$}。}
    \IT{\CNS*{BOOLE-ANDC2}} {\retval{$\text{\VAR{int-a}} \land \lnot\text{\VAR{int-b}}$}。}
    \IT{\CNS*{BOOLE-NAND}} {\retval{$\lnot(\text{\VAR{int-a}} \land \text{\VAR{int-b}})$}。}
    \IT{\CNS*{BOOLE-IOR}\qquad\qquad} {\retval{$\text{\VAR{int-a}} \lor \text{\VAR{int-b}}$}。}
    \IT{\CNS*{BOOLE-ORC1}\qquad} {\retval{$\lnot \text{\VAR{int-a}} \lor \text{\VAR{int-b}}$}。}
    \IT{\CNS*{BOOLE-ORC2}\qquad} {\retval{$\text{\VAR{int-a}} \lor \lnot\text{\VAR{int-b}}$}。}
    \IT{\CNS*{BOOLE-XOR}\qquad\qquad} {\retval{$\lnot(\text{\VAR{int-a}} \equiv \text{\VAR{int-b}})$}。}
    \IT{\CNS*{BOOLE-NOR}\qquad\qquad} {\retval{$\lnot(\text{\VAR{int-a}} \lor \text{\VAR{int-b}})$}。}
  \end{LIST}

  \IT{(\FU*{LOGNOT}\VAR{ integer})\qquad\qquad}
  {
    \retval{$\lnot\text{\VAR{integer}}$}。
  }

  \IT{\arrGOO{(\FU*{LOGEQV} \OPn{\VAR{ integer}})\\
      (\FU*{LOGAND} \OPn{\VAR{ integer}})}{.}}
  {
  返回 \retval{\VAR{integer}s 的 按位同或 或 按位与}。无
  \VAR{integer} 时返回 \retval{$-1$}。
  }

  \IT{(\FU*{LOGANDC1} \VAR{int-a} \VAR{int-b})}
  {
    \retval{$\lnot \text{\VAR{int-a}} \land \text{\VAR{int-b}}$}。
   }

  \IT{(\FU*{LOGANDC2} \VAR{int-a} \VAR{int-b})}
  {
    \retval{$\text{\VAR{int-a}} \land \lnot\text{\VAR{int-b}}$}。
  }

  \IT{(\FU*{LOGNAND} \VAR{int-a} \VAR{int-b})\qquad}
  {
  \retval{$\lnot(\text{\VAR{int-a}} \land \text{\VAR{int-b}})$}。
  }

  \IT{\arrGOO{(\FU*{LOGXOR} \OPn{\VAR{ integer}})\\
      (\FU*{LOGIOR} \OPn{\VAR{ integer}})}{.}}
  {
  返回 \retval{\VAR{integer}s 的 按位异或 或 按位或}。无
  \VAR{integer} 时返回 \retval{0}。
  }

  \IT{(\FU*{LOGORC1} \VAR{int-a} \VAR{int-b})}
  {
    \retval{$\lnot \text{\VAR{int-a}} \lor \text{\VAR{int-b}}$}。
 }

  \IT{(\FU*{LOGORC2} \VAR{int-a} \VAR{int-b})}
  {
    \retval{$\text{\VAR{int-a}} \lor \lnot\text{\VAR{int-b}}$}。
  }

  \IT{(\FU*{LOGNOR} \VAR{int-a} \VAR{int-b})}
  {
    \retval{$\lnot(\text{\VAR{int-a}} \lor \text{\VAR{int-b}})$}。
  }

  \IT{(\FU*{LOGBITP} \VAR{i} \VAR{int})}
  {
    当 \VAR{int} 的零索引第 \VAR{i} 位为 1 时为 \retval{\T}。
  }

  \IT{(\FU*{LOGTEST} \VAR{int-a} \VAR{int-b})}
  {当 \VAR{int-a} 与 \VAR{int-b} 中存在相同位皆为 1 时返回
  \retval{\T}。
  }

  \IT{(\FU*{LOGCOUNT} \VAR{int})}
  {
    $\text{\VAR{int}}\ge 0$ 时为 \retval{1 的数目},$\text{\VAR{int}}< 0$
    时为 \retval{0 的数目}。
  }


\end{LIST}

%%%%%%%%%%%%%%%%%%%%%%%%%%%%%%%%%%%%%%%%%%%%%%%%%%
\subsection{整数函数} 
%%%%%%%%%%%%%%%%%%%%%%%%%%%%%%%%%%%%%%%%%%%%%%%%%%
\begin{LIST}{1cm}

  \IT{(\FU*{INTEGER-LENGTH} \VAR{integer})}
  {
  \VAR{integer} 的 \retval{比特位数}。
  }

  \IT{(\FU*{LDB-TEST} \VAR{byte-spec} \VAR{integer})}
  {
  当 \VAR{integer} 在 \VAR{byte-spec}
  指定的位上为 1 时返回 \retval{\T}。
  }

  \IT{(\FU*{ASH} \VAR{integer} \VAR{count})}
  {
  返回 \retval{\VAR{integer}}
  副本的左移结果,以零补齐,当
  $\VAR{count}<0$
  时,右移并丢弃多余的位。
  }

  \IT{(\FU*{LDB} \VAR{byte-spec} \VAR{integer})}
  {
  从 \VAR{integer} 中提取 \VAR{byte-spec} 指定的
  \retval{byte}。可 \kwd{setf}。
  }

  \IT{(\xorGOO{\FU*{DEPOSIT-FIELD}\\
      \FU*{DPB}}{\}}
    \VAR{int-a} \VAR{byte-spec} \VAR{int-b})}
  {
  返回 \retval{\VAR{int-b}},其中 \VAR{byte-spec}
  指定的位用 \VAR{int-a} 相应的位,或用 \VAR{int-a} 的低 (\FU{byte-size}
  \VAR{byte-spec}) 位替换。
  }

  \IT{(\FU*{MASK-FIELD} \VAR{byte-spec} \VAR{integer})}
  {
  返回 \retval{\VAR{integer}} 副本,其中除 \VAR{byte-spec}
  指定的位外全为 0。可 \kwd{setf}。
  }

  \IT{(\FU*{BYTE} \VAR{size} \VAR{position})}
  {
  \retval{字节说明符},指定从
  \VAR{position} 起的 \VAR{size} 位。
  }

  \IT{\arrGOO{(\FU*{BYTE-SIZE} \VAR{ byte-spec})\\
      (\FU*{BYTE-POSITION} \VAR{ byte-spec})}{.}}
  {
  \VAR{byte-spec} 的 \retval{大小} 或 \retval{位置}。
  }

\end{LIST}


%%%%%%%%%%%%%%%%%%%%%%%%%%%%%%%%%%%%%%%%%%%%%%%%%%
\subsection{特定于实现} 
%%%%%%%%%%%%%%%%%%%%%%%%%%%%%%%%%%%%%%%%%%%%%%%%%%
\begin{LIST}{1cm}

  \IT{\arrGOO{\CNS{SHORT-FLOAT}\\
      \CNS{SINGLE-FLOAT}\\
      \CNS{DOUBLE-FLOAT}\\
      \CNS{LONG-FLOAT}}{\}}\kwd{-}%
    \xorGOO{\kwd{EPSILON}\\
      \kwd{NEGATIVE-EPSILON}}{.}}
  {
  \index{SHORT-FLOAT-EPSILON}%
  \index{SINGLE-FLOAT-EPSILON}%
  \index{DOUBLE-FLOAT-EPSILON}%
  \index{LONG-FLOAT-EPSILON}%
  \index{SHORT-FLOAT-NEGATIVE-EPSILON}%
  \index{SINGLE-FLOAT-NEGATIVE-EPSILON}%
  \index{DOUBLE-FLOAT-NEGATIVE-EPSILON}%
  \index{LONG-FLOAT-NEGATIVE-EPSILON}%
  数字加减时的最小分度值。
  }

  \IT{\arrGOO{%
      \CNS{LEAST-NEGATIVE}\\
      \CNS{LEAST-NEGATIVE-NORMALIZED}\\
      \CNS{LEAST-POSITIVE}\\
      \CNS{LEAST-POSITIVE-NORMALIZED}}{\}}%
    \kwd{-}%
    \xorGOO{%
      \kwd{SHORT-FLOAT}\\
      \kwd{SINGLE-FLOAT}\\
      \kwd{DOUBLE-FLOAT}\\
      \kwd{LONG-FLOAT}}{.}}
  {
  \index{LEAST-NEGATIVE-SHORT-FLOAT}%
  \index{LEAST-NEGATIVE-NORMALIZED-SHORT-FLOAT}%
  \index{LEAST-NEGATIVE-SINGLE-FLOAT}%
  \index{LEAST-NEGATIVE-NORMALIZED-SINGLE-FLOAT}%
  \index{LEAST-NEGATIVE-DOUBLE-FLOAT}%
  \index{LEAST-NEGATIVE-NORMALIZED-DOUBLE-FLOAT}%
  \index{LEAST-NEGATIVE-LONG-FLOAT}%
  \index{LEAST-NEGATIVE-NORMALIZED-LONG-FLOAT}%
  \index{LEAST-POSITIVE-SHORT-FLOAT}%
  \index{LEAST-POSITIVE-NORMALIZED-SHORT-FLOAT}%
  \index{LEAST-POSITIVE-SINGLE-FLOAT}%
  \index{LEAST-POSITIVE-NORMALIZED-SINGLE-FLOAT}%
  \index{LEAST-POSITIVE-DOUBLE-FLOAT}%
  \index{LEAST-POSITIVE-NORMALIZED-DOUBLE-FLOAT}%
  \index{LEAST-POSITIVE-LONG-FLOAT}%
  \index{LEAST-POSITIVE-NORMALIZED-LONG-FLOAT}%
  最接近 $-0$ 或 $+0$ 的数字。
  }

  \IT{\arrGOO{\CNS{MOST-NEGATIVE}\\
      \CNS{MOST-POSITIVE}}{\}}%
    \kwd{-}%
    \xorGOO{%
      \kwd{SHORT-FLOAT}\\
      \kwd{SINGLE-FLOAT}\\
      \kwd{DOUBLE-FLOAT}\\
      \kwd{LONG-FLOAT}\\
      \kwd{FIXNUM}}{.}}
  {
  \index{MOST-NEGATIVE-DOUBLE-FLOAT}%
  \index{MOST-NEGATIVE-LONG-FLOAT}%
  \index{MOST-NEGATIVE-SHORT-FLOAT}%
  \index{MOST-NEGATIVE-SINGLE-FLOAT}%
  \index{MOST-NEGATIVE-FIXNUM}%
  \index{MOST-POSITIVE-DOUBLE-FLOAT}%
  \index{MOST-POSITIVE-LONG-FLOAT}%
  \index{MOST-POSITIVE-SHORT-FLOAT}%
  \index{MOST-POSITIVE-SINGLE-FLOAT}%
  \index{MOST-POSITIVE-FIXNUM}%
  最接近 $-\infty$ 或 $+\infty$ 的数字。
  }

  \IT{\arrGOO{(\FU*{DECODE-FLOAT} \VAR{ n})\\
      (\FU*{INTEGER-DECODE-FLOAT} \VAR{ n})}{.}}
  {
  返回 \kwd{float} \VAR{n} 的
  \retval{尾数}、\retvalii{指数}及\retvaliii{符号}。
  }

  \IT{(\FU*{SCALE-FLOAT} \VAR{n} \Op{\VAR{i}})}
  {
  用 \VAR{n} 的基数 $b$ 返回 $n b^{i}$。
  }

  \IT{\arrGOO{
      (\FU*{FLOAT-RADIX} \VAR{ n})\\
      (\FU*{FLOAT-DIGITS} \VAR{ n})\\
      (\FU*{FLOAT-PRECISION} \VAR{ n})}{.}}
  {
  float \VAR{n} 的
  \retval{基数}、\retval{位数} 或 \retval{精度}。
  }

  \IT{(\FU*{UPGRADED-COMPLEX-PART-TYPE} \VAR{foo} \Op{\VAR{environment}\DF{\NIL}})}
  {可持有类型 \VAR{foo} 的最具体
    \kwd{complex} \retval{类型}。
  }

\end{LIST}


% LocalWords:  de na der nored ored

%%% Local Variables: 
%%% mode: latex
%%% TeX-master: "clqr"
%%% End: 
