% Copyright (C) 2008, 2009, 2010, 2011, 2012, 2014 Bert Burgemeister
%
% Permission is granted to copy, distribute and/or modify this
% document under the terms of the GNU Free Documentation License,
% Version 1.2; with no Invariant Sections, no Front-Cover Texts and
% no Back-Cover Texts. For details see file COPYING.
%

%%%%%%%%%%%%%%%%%%%%%%%%%%%%%%%%%%%%%%%%%%%%%%%%%%
\section{数组} 
%%%%%%%%%%%%%%%%%%%%%%%%%%%%%%%%%%%%%%%%%%%%%%%%%%
\label{section:数组}

%%%%%%%%%%%%%%%%%%%%%%%%%%%%%%%%%%%%%%%%%%%%%%%%%%
\subsection{谓词} 
%%%%%%%%%%%%%%%%%%%%%%%%%%%%%%%%%%%%%%%%%%%%%%%%%%
\begin{LIST}{1cm}

  \IT{\arrGOO{%
      (\FU*{ARRAYP} \VAR{ foo})\\
      (\FU*{VECTORP} \VAR{ foo})\\
      (\FU*{SIMPLE-VECTOR-P} \VAR{ foo})\\
      (\FU*{BIT-VECTOR-P} \VAR{ foo})\\
      (\FU*{SIMPLE-BIT-VECTOR-P} \VAR{ foo})}{.}}
  {
  当 \VAR{foo} 为指定类型时为 \retval{\T}。
  }

  \IT{\arrGOO{(\FU*{ADJUSTABLE-ARRAY-P} \VAR{ array})\\
      (\FU*{ARRAY-HAS-FILL-POINTER-P} \VAR{ array})}{.}}
  {
  当 \VAR{array} 可变长/有填充指针时为
  \retval{\T}。
  }

  \IT{(\FU*{ARRAY-IN-BOUNDS-P} \VAR{array} \Op{\VAR{subscripts}})}
  {
  当下标在数组边界内时返回 \retval{\T}。
  }
  % No default subscripts in standard.

\end{LIST}


%%%%%%%%%%%%%%%%%%%%%%%%%%%%%%%%%%%%%%%%%%%%%%%%%%
\subsection{数组函数} 
%%%%%%%%%%%%%%%%%%%%%%%%%%%%%%%%%%%%%%%%%%%%%%%%%%
\begin{LIST}{1cm}

  \IT{(\xorGOO{%
      \FU*{MAKE-ARRAY}\VAR{ dimension-sizes }
      \OP{\kwd{:adjustable} \VAR{ bool}\DF{\NIL}}\\
      \FU*{ADJUST-ARRAY } \DES{\VAR{array}}
      \VAR{ dimension-sizes}}{\}}
    \orGOO{\kwd{:element-type} \VAR{ type}\DF{\T}\\
      \kwd{:fill-pointer } \Goo{\VAR{num}\XOR\VAR{bool}}\DF{\NIL}\\
      \xorGOO{\kwd{:initial-element} \VAR{ obj}\\
        \kwd{:initial-contents} \VAR{ tree-or-array}\\
        \kwd{:displaced-to } \VAR{array}\DF{\NIL}\text{ }
        \Op{\kwd{:displaced-index-offset } \VAR{i}\DF{\LIT{0}}}}{.}}{\}})}
  { 
  返回新的或可变长的 \retval{向量} 或 \retval{数组}。
  }

  \IT{(\FU*{AREF} \VAR{array} \OP{\VAR{subscripts}})}
  {
  返回
  \VAR{subscripts} 指向的 \retval{数组元素}。可 \kwd{setf}。
  }
  % No subscripts means sole element of rank zero array.

  \IT{(\FU*{ROW-MAJOR-AREF} \VAR{array} \VAR{i})}
  {
  返回以行优先顺序的 \VAR{array}
  \retval{第 \VAR{i} 个元素}。可 \kwd{setf}。
  }

  \IT{(\FU*{ARRAY-ROW-MAJOR-INDEX} \VAR{array} \Op{\VAR{subscripts}})}
  {
  \VAR{subscripts} 所指定元素的行优先顺序 \retval{索引}。
  }

  \IT{(\FU*{ARRAY-DIMENSIONS} \VAR{array})}
  {
  包含 \VAR{array} 每维度长度的 \retval{列表}。
  }

  \IT{(\FU*{ARRAY-DIMENSION} \VAR{array} \VAR{i})}
  {
  \VAR{array} \retval{第 \VAR{i} 维的长度}。
  }

  \IT{(\FU*{ARRAY-TOTAL-SIZE} \VAR{array})}
  {
    \VAR{array} 中的 \retval{元素数}。
  }

  \IT{(\FU*{ARRAY-RANK} \VAR{array})}
  {
    \VAR{array} 的 \retval{维数}。
  }

  \IT{(\FU*{ARRAY-DISPLACEMENT} \VAR{array})}
  {
  \retval{目标数组} 及 \retvalii{偏移量}。
  }

  \IT{\arrGOO{(\FU*{BIT} \VAR{ bit-array } \Op{\VAR{subscripts}})\\
      (\FU*{SBIT} \VAR{ simple-bit-array } \Op{\VAR{subscripts}})}{.}}
  {
  返回 \VAR{bit-array} 或 \VAR{simple-bit-array} 的
  \retval{element}。可 \kwd{setf}。
  }

  \IT{(\FU*{BIT-NOT} \DES{\VAR{bit-array}} \Op{\DES{\VAR{result-bit-array}}\DF{\NIL}})}
  {
  返回 \VAR{bit-array} 的按位反 \retval{结果}。若 \VAR{result-bit-array} 为 \T,将结果放入
  \VAR{bit-array},若为
  \NIL,创建新的结果数组。
  }

  \IT{(\xorGOO{%
      \FU*{BIT-EQV}\\
      \FU*{BIT-AND}\\
      \FU*{BIT-ANDC1}\\
      \FU*{BIT-ANDC2}\\
      \FU*{BIT-NAND}\\
      \FU*{BIT-IOR}\\
      \FU*{BIT-ORC1}\\
      \FU*{BIT-ORC2}\\
      \FU*{BIT-XOR}\\
      \FU*{BIT-NOR}}{\}} \DES{\VAR{bit-array-a}} \VAR{bit-array-b}
    \Op{\DES{\VAR{result-bit-array}}\DF{\NIL}})}
  {
  返回 \VAR{bit-array-a} 与 \VAR{bit-array-b}
  的按位逻辑运算结果(辨析:第 \pageref{section:逻辑函数}
  页 \FU{boole} 操作)。若 \VAR{result-bit-array} 为 \T,将结果放入
  \VAR{bit-array-a},若为
  \NIL,创建新的结果数组。
  }

  \IT{\CNS*{ARRAY-RANK-LIMIT}}
  {
  数组阶数上界,$\geq 8$。
  }

  \IT{\CNS*{ARRAY-DIMENSION-LIMIT}}
  {
  数组维数上界,$\geq 1024$。
  }

  \IT{\CNS*{ARRAY-TOTAL-SIZE-LIMIT}}
  {
  数组大小上界,$\geq 1024$。
  }

\end{LIST}


%%%%%%%%%%%%%%%%%%%%%%%%%%%%%%%%%%%%%%%%%%%%%%%%%%
\subsection{向量函数} 
%%%%%%%%%%%%%%%%%%%%%%%%%%%%%%%%%%%%%%%%%%%%%%%%%%

序列函数也可用于操作向量,参见第
\ref{section:序列} 章。

\begin{LIST}{1cm}
  
  \IT{(\FU*{VECTOR} \OPn{\VAR{foo}})}
  {
    返回新的 \retval{\VAR{foo}s 的简单向量}。
  }
  % No default foo in standard.

  \IT{(\FU*{SVREF} \VAR{vector} \VAR{i})}
  {
    简单 \VAR{vector} 的 \retval{第\ \VAR{i} 个元素}。可 \kwd{setf}。
  }

  \IT{(\FU*{VECTOR-PUSH} \VAR{foo} \DES{\VAR{vector}})}
  {
  若 \VAR{vector} 的填充指针等于
  \VAR{vector} 大小,返回 \retval{\NIL}。
  否则,以 \VAR{foo} 替换 \retval{填充指针}
  指向的 \VAR{vector} 元素,然后增加填充指针。
  }

  \IT{(\FU*{VECTOR-PUSH-EXTEND} \VAR{foo} \DES{\VAR{vector}}
    \Op{\VAR{num}})}
  {
  以 \VAR{foo} 替换 \retval{填充指针}
  指向的 \VAR{vector} 元素,然后增加填充指针。如有必要,将向量大小扩大
  $\ge \VAR{num}$。
  }

  \IT{(\FU*{VECTOR-POP} \DES{\VAR{vector}})}
  {
  填充指针减 1,然后返回指向的
  \retval{\VAR{vector} 元素}。
  }

  \IT{(\FU*{FILL-POINTER} \VAR{vector})}
  {
  \VAR{vector} 的 \retval{填充指针}。可 \kwd{setf}。
  }

\end{LIST}



%%% Local Variables: 
%%% mode: latex
%%% TeX-master: "clqr"
%%% End: 
