% Copyright (C) 2008, 2010, 2014 Bert Burgemeister
%
% Permission is granted to copy, distribute and/or modify this
% document under the terms of the GNU Free Documentation License,
% Version 1.2; with no Invariant Sections, no Front-Cover Texts and
% no Back-Cover Texts. For details see file COPYING.
%

%%%%%%%%%%%%%%%%%%%%%%%%%%%%%%%%%%%%%%%%%%%%%%%%%%
\section{字符串} 
%%%%%%%%%%%%%%%%%%%%%%%%%%%%%%%%%%%%%%%%%%%%%%%%%%
数组和序列函数也可用于操作字符串,参见第
\pageref{section:数组} 页和第 \pageref{section:序列} 页。

\begin{LIST}{1cm}


  \IT{\arrGOO{(\FU*{STRINGP} \VAR{ foo})\\
      (\FU*{SIMPLE-STRING-P} \VAR{ foo})}{.}}
  {
    当 \VAR{foo} 为指定类型时为 \retval{\T}。
  }

  \IT{(\xorGOO{\FU*{STRING=}\\\FU*{STRING-EQUAL}}{\}} \VAR{foo}
    \VAR{bar} 
    \orGOO{\kwd{:start1} \VAR{ start-foo}\DF{\LIT{0}}\\
      \kwd{:start2} \VAR{ start-bar}\DF{\LIT{0}}\\
      \kwd{:end1} \VAR{ end-foo}\DF{\NIL}\\
      \kwd{:end2} \VAR{ end-bar}\DF{\NIL}}{\}})}
  {
    当 \VAR{foo} 和 \VAR{bar} 的子序列相等时返回
    \retval{\T}。分别检查/忽略大小写。
  }

  \IT{(\xorGOO{%
      \FU{STRING}\Goo{\kwd{/= }\XOR\kwd{-NOT-EQUAL}}\\
      \FU{STRING}\Goo{\kwd{\boldmath$>$ }\XOR\kwd{-GREATERP}}\\
      \FU{STRING}\Goo{\kwd{\boldmath$>$= }\XOR\kwd{-NOT-LESSP}}\\
      \FU{STRING}\Goo{\kwd{\boldmath$<$ }\XOR\kwd{-LESSP}}\\
      \FU{STRING}\Goo{\kwd{\boldmath$<$= }\XOR\kwd{-NOT-GREATERP}}}{\}}
    \VAR{foo} \VAR{bar}
    \orGOO{\kwd{:start1} \VAR{ start-foo}\DF{\LIT{0}}\\
      \kwd{:start2} \VAR{ start-bar}\DF{\LIT{0}}\\
      \kwd{:end1} \VAR{ end-foo}\DF{\NIL}\\
      \kwd{:end2} \VAR{ end-bar}\DF{\NIL}}{\}})}
  {\index{STRING/=}\index{STRING-NOT-EQUAL}%
    \index{STRING>@STRING$>$}\index{STRING-GREATERP}%
    \index{STRING>=@STRING$>$=}\index{STRING-NOT-LESSP}%
    \index{STRING<@STRING$<$}\index{STRING-LESSP}%
    \index{STRING<=@STRING$<$=}\index{STRING-NOT-GREATERP}%
    测试 \VAR{foo} 的字典序是否不等于、大于、不小于、小于、不大于
    \VAR{bar},返回在 \VAR{foo} 第一次命中的
    \retval{位置},否则返回
    \retval{\NIL}。分别检查/忽略大小写。
  }

  \IT{(\FU*{MAKE-STRING} \VAR{size} \orGOO{\kwd{:initial-element} \VAR{ char}\\
      \kwd{:element-type} \VAR{ type}\DF{\kwd{character}}}{\}})}
  {
  返回长度为 \VAR{size} 的 \retval{字符串}。
  }

  \IT{\arrGOO{%
      (\FU*{STRING} \VAR{ x})\\
      (\xorGOO{%
        \FU*{STRING-CAPITALIZE}\\
        \FU*{STRING-UPCASE}\\
        \FU*{STRING-DOWNCASE}}{\}}
      \VAR{ x }  
      \orGOO{\kwd{:start} \VAR{ start}\DF{\LIT{0}}\\
        \kwd{:end} \VAR{ end}\DF{\NIL}}{\}})}{.}}
  {
    将
    \VAR{x}(\kwd{symbol}、\kwd{string}、\kwd{character})转换为
    \retval{字符串}、\retval{单词首字母大写字符串}、\retval{全大写字符串}、\retval{全小写字符串}。
  }

  \IT{(\xorGOO{%
      \FU*{NSTRING-CAPITALIZE}\\
      \FU*{NSTRING-UPCASE}\\
      \FU*{NSTRING-DOWNCASE}}{\}}
    \VAR{\DES{string}}  
    \orGOO{\kwd{:start} \VAR{ start}\DF{\LIT{0}}\\
      \kwd{:end} \VAR{ end}\DF{\NIL}}{\}})}
  {
    将 \VAR{string}
    转换为
    \retval{单词首字母大写字符串}、\retval{全大写字符串}、\retval{全小写字符串}。
  }

  \IT{(\xorGOO{\FU*{STRING-TRIM}\\
      \FU*{STRING-LEFT-TRIM}\\
      \FU*{STRING-RIGHT-TRIM}}{\}} \VAR{char-bag} \VAR{string})}
  {
    从两端、从头、从尾除去序列 \VAR{char-bag}
    中的字符,并返回
    \retval{\VAR{string}}。
  }

  \IT{\arrGOO{(\FU*{CHAR} \VAR{ string} \VAR{ i})\\
      (\FU*{SCHAR} \VAR{ string} \VAR{ i})}{.}}
  {
    忽略/检查填充指针,返回字符串中零索引的
    \retval{第\ \VAR{i} 个字符}。可 \kwd{setf}。
  }

  \IT{(\FU*{PARSE-INTEGER} \VAR{string}
    \orGOO{\kwd{:start} \VAR{ start}\DF{\LIT{0}}\\
      \kwd{:end} \VAR{ end}\DF{\NIL}\\
      \kwd{:radix} \VAR{ int}\DF{\LIT{10}}\\
      \kwd{:junk-allowed} \VAR{ bool}\DF{\NIL}}{\}})}
  {
    从 \VAR{string} 中解析并返回
    \retval{整数} 和结束时的 \retvalii{索引}。
  }

\end{LIST}

%%% Local Variables: 
%%% mode: latex
%%% TeX-master: "clqr"
%%% End: 
