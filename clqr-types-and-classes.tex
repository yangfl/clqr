% Copyright (C) 2008, 2009, 2010, 2011, 2012, 2014 Bert Burgemeister
%
% Permission is granted to copy, distribute and/or modify this
% document under the terms of the GNU Free Documentation License,
% Version 1.2 or any later version published by the Free Software
% Foundation; with no Invariant Sections, no Front-Cover Texts and
% no Back-Cover Texts. For details see file COPYING.
%

%%%%%%%%%%%%%%%%%%%%%%%%%%%%%%%%%%%%%%%%%%%%%%%%%%
\section{类型及类} 
%%%%%%%%%%%%%%%%%%%%%%%%%%%%%%%%%%%%%%%%%%%%%%%%%%

\begin{figure}
  \begin{center}
    \begin{sideways}
      \includegraphics{clqr-types-and-classes.1}
    \end{sideways}
  \end{center}
  \caption[]{
    系统类(\includegraphics{
    clqr-types-and-classes.3})、类(\includegraphics{
    clqr-types-and-classes.4})、类型(\includegraphics{
    clqr-types-and-classes.2})、状况类型(\includegraphics{
    clqr-types-and-classes.5})的优先级顺序。\\
    每种类型也是空类型 \NIL 的超类型。
    \label{data-types}%
    %
    \index{*@\A}%
    \index{T}%
    \index{ATOM}%
    \index{READTABLE}%
    \index{PACKAGE}%
    \index{SYMBOL}%
    \index{KEYWORD}%
    \index{BOOLEAN}%
    \index{RESTART}%
    \index{RANDOM-STATE}%
    \index{HASH-TABLE}%
    \index{STRUCTURE-OBJECT}%
    \index{STANDARD-OBJECT}%
    \index{NULL}%
    \index{CLASS}%
    \index{BUILT-IN-CLASS}%
    \index{STANDARD-CLASS}%
    \index{STRUCTURE-CLASS}%
    \index{METHOD}%
    \index{STANDARD-METHOD}%
    \index{METHOD-COMBINATION}%
    \index{CHARACTER}%
    \index{FUNCTION}%
    \index{COMPILED-FUNCTION}%
    \index{GENERIC-FUNCTION}%
    \index{STANDARD-GENERIC-FUNCTION}%
    \index{PATHNAME}%
    \index{LOGICAL-PATHNAME}%
    \index{NUMBER}%
    \index{COMPLEX}%
    \index{REAL}%
    \index{FLOAT}%
    \index{SHORT-FLOAT}%
    \index{SINGLE-FLOAT}%
    \index{DOUBLE-FLOAT}%
    \index{LONG-FLOAT}%
    \index{RATIONAL}%
    \index{INTEGER}%
    \index{RATIO}%
    \index{SIGNED-BYTE}%
    \index{FIXNUM}%
    \index{BIGNUM}%
    \index{UNSIGNED-BYTE}%
    \index{BIT}%
    \index{LIST}%
    \index{SEQUENCE}%
    \index{CONS}%
    \index{ARRAY}%
    \index{SIMPLE-ARRAY}%
    \index{VECTOR}%
    \index{STRING}%
    \index{SIMPLE-STRING}%
    \index{BASE-STRING}%
    \index{SIMPLE-BASE-STRING}%
    \index{SIMPLE-VECTOR}%
    \index{BIT-VECTOR}%
    \index{SIMPLE-BIT-VECTOR}%
    \index{STREAM}%
    \index{FILE-STREAM}%
    \index{TWO-WAY-STREAM}%
    \index{SYNONYM-STREAM}%
    \index{STRING-STREAM}%
    \index{BROADCAST-STREAM}%
    \index{CONCATENATED-STREAM}%
    \index{ECHO-STREAM}%
    \index{EXTENDED-CHAR}%
    \index{BASE-CHAR}%
    \index{STANDARD-CHAR}%
    \index{CONDITION}%
    \index{SERIOUS-CONDITION}%
    \index{STORAGE-CONDITION}%
    \index{SIMPLE-TYPE-ERROR}%
    \index{TYPE-ERROR}%
    \index{ERROR}%
    \index{PROGRAM-ERROR}%
    \index{CONTROL-ERROR}%
    \index{PACKAGE-ERROR}%
    \index{PRINT-NOT-READABLE}%
    \index{STREAM-ERROR}%
    \index{PARSE-ERROR}%
    \index{CELL-ERROR}%
    \index{FILE-ERROR}%
    \index{ARITHMETIC-ERROR}%
    \index{SIMPLE-CONDITION}%
    \index{WARNING}%
    \index{STYLE-WARNING}%
    \index{SIMPLE-ERROR}%
    \index{SIMPLE-WARNING}%
    \index{UNDEFINED-FUNCTION}%
    \index{UNBOUND-VARIABLE}%
    \index{UNBOUND-SLOT}%
    \index{END-OF-FILE}%
    \index{READER-ERROR}%
    \index{DIVISION-BY-ZERO}%
    \index{FLOATING-POINT-INEXACT}%
    \index{FLOATING-POINT-OVERFLOW}%
    \index{FLOATING-POINT-UNDERFLOW}%
    \index{FLOATING-POINT-INVALID-OPERATION}%
  }
\end{figure}

对任何类,总有对应的同名类型。

\begin{LIST}{1cm}

  \IT{(\FU*{TYPEP} \VAR{foo} \VAR{type} \Op{\VAR{environment}\DF{\NIL}})}
  {
    当 \VAR{foo} 为 \VAR{type} 时为 \retval{\T}。
  }

  \IT{(\FU*{SUBTYPEP} \VAR{type-a} \VAR{type-b}
    \Op{\VAR{environment}})}
  {
    若 \VAR{type-a} 为 \VAR{type-b}
    的可识别子类型,返回 \retval{\T},若关系无法确定,返回
    \retvalii{\NIL}。
  }

  \IT{(\SO*{THE} \NEV{\VAR{type}} \VAR{form})}
  {
    声明 \retval{\VAR{form} 的所有值} 为
    \VAR{type}。
  }

  \IT{(\FU*{COERCE} \VAR{object} \VAR{type})}
  {
    将 \retval{\VAR{object}} 强制转换为 \VAR{type}。
  }

  \IT{(\MC*{TYPECASE} \VAR{foo} \OPn{(\NEV{\VAR{type}} \PROGN{\VAR{a-form}})} 
    \OP{(\xorGOO{\kwd*{OTHERWISE}\\
        \T}{\}} \PROGN{\VAR{b-form}\DF{\NIL}})})}
  {
    返回第一个其 \VAR{type} 为 \VAR{foo} 的
    \retval{\OPn{\VAR{a-form}} 的所有值}。若无 \VAR{type}
    匹配,返回 \retval{\OPn{\VAR{b-form}s} 的所有值}。
    % Keep \OPn{\VAR{a-form}} instead of the otherwise preferable
    % \VAR{a-form}s because we're talking about one set of a-forms out
    % of several.
  }

  \IT{(\xorGOO{%
      \MC*{ETYPECASE}\\
      \MC*{CTYPECASE}}{\}}
    \VAR{foo}
    \OPn{(\NEV{\VAR{type}} \PROGN{\VAR{form}})})}
  {
    返回第一个其 \VAR{type} 为 \VAR{foo} 的
    \retval{\OPn{\VAR{a-form}} 的所有值}。若无 \VAR{type}
    匹配,产生不可纠正/可纠正 \kwd{type-error}。
    % Keep \OPn{\VAR{form}} instead of the otherwise preferable
    % \VAR{form}s because we're talking about one set of forms out
    % of several.
  }

  \IT{(\FU*{TYPE-OF} \VAR{foo})}
  {
    \retval{\VAR{foo} 的类型}。
  }

  \IT{(\MC*{CHECK-TYPE} \VAR{place} \VAR{type}
    \Op{\VAR{string}\DF{\Goo{\LIT{a}\XOR\LIT{an}}\hspace{.3ex}\VAR{type}}})}
  {
    若 \VAR{place} 非 \VAR{type},产生可纠正
    \kwd{type-error}。返回 \retval{\NIL}。
  }

  \IT{(\FU*{STREAM-ELEMENT-TYPE} \VAR{stream})}
  {
    \VAR{stream} 对象的 \retval{类型}。
  }

  \IT{(\FU*{ARRAY-ELEMENT-TYPE} \VAR{array})}
  {
  \VAR{array} 可包含的元素 \retval{类型}。
  }

  \IT{(\FU*{UPGRADED-ARRAY-ELEMENT-TYPE} \VAR{type}
    \Op{\VAR{environment}\DF{\NIL}})}
  {
  可包含 \VAR{type} 元素的最具体数组的
  \retval{元素类型}。
  }

  \IT{(\MC*{DEFTYPE} \VAR{foo} (\OPn{\VAR{macro-$\lambda$}})
    \OPn{(\kwd{declare} \OPn{\NEV{\VAR{decl}}})} \Op{\NEV{\VAR{doc}}} 
    \PROGN{\VAR{form}})}
  {
  定义类型 \retval{\VAR{foo}},当以 (\VAR{foo}
  \OPn{\NEV{\VAR{arg}}})(或在 \VAR{macro-$\lambda$}
  包含任何必要参数时以 \VAR{foo})引用时,将展开的 \VAR{form}s 应用至 \VAR{arg}s
  并返回新类型。对于 (\OPn{\VAR{macro-$\lambda$}}) 参见第
  \pageref{section:宏} 页,但默认值为 \kwd{\A},而不是
  \NIL。\VAR{form}s 包含于名为 \VAR{foo}
  的隐式 \SO{block} 中。
  }

  \IT{\arrGOO{(\kwd*{EQL } \VAR{foo})\\
      (\kwd*{MEMBER } \OPn{\VAR{foo}})}{.}}
  {
  构成 \VAR{foo} 或 \VAR{foo}s 的类型的说明符。
  }

  \IT{(\kwd*{SATISFIES} \VAR{predicate})}
  {
  所有满足 \VAR{predicate} 对象的类型说明符。
  }

  \IT{(\kwd*{MOD} \VAR{n})}
  {
  所有 $<n$ 非负整数的类型说明符。
  }

  \IT{(\kwd*{NOT} \VAR{type})}
  {
  \VAR{type} 的补集。
  }

  \IT{(\kwd*{AND} \OPn{\VAR{type}}\DF{\T})}
  {
  \VAR{type}s 交集的类型说明符。
  }

  \IT{(\kwd*{OR} \OPn{\VAR{type}}\DF{\NIL})}
  {
  \VAR{type}s 并集的类型说明符。
  }

  \IT{(\kwd*{VALUES} \OPn{\VAR{type}} \OP{\OPn{\kwd{\&optional} \VAR{type}}
      \Op{\kwd{\&rest} \VAR{other-args}}})}
  {
  多值的类型说明符。
  }

  \IT{\kwd{\A}}
  {
    \index{*@\A}%
    作为类型参数(参照图 \ref{data-types}):无限制。
  }
  \end{LIST}



%%% Local Variables: 
%%% mode: latex
%%% TeX-master: "clqr"
%%% End: 
