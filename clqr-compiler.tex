% Copyright (C) 2008, 2009, 2010, 2011, 2012, 2014 Bert Burgemeister
%
% Permission is granted to copy, distribute and/or modify this
% document under the terms of the GNU Free Documentation License,
% Version 1.2; with no Invariant Sections, no Front-Cover Texts and
% no Back-Cover Texts. For details see file COPYING.
%

%%%%%%%%%%%%%%%%%%%%%%%%%%%%%%%%%%%%%%%%%%%%%%%%%%
\section{编译器} 
%%%%%%%%%%%%%%%%%%%%%%%%%%%%%%%%%%%%%%%%%%%%%%%%%%

%%%%%%%%%%%%%%%%%%%%%%%%%%%%%%%%%%%%%%%%%%%%%%%%%%
\subsection{谓词} 
%%%%%%%%%%%%%%%%%%%%%%%%%%%%%%%%%%%%%%%%%%%%%%%%%%

\begin{LIST}{1cm}

  \IT{(\FU*{SPECIAL-OPERATOR-P} \VAR{foo})}
  {
  当 \VAR{foo} 为特殊操作符时为 \retval{\T}。
  }

  \IT{(\FU*{COMPILED-FUNCTION-P} \VAR{foo})}
  {
  当 \VAR{foo} 为类型 \kwd{compiled-function} 时为 \retval{\T}。
  }

\end{LIST}


%%%%%%%%%%%%%%%%%%%%%%%%%%%%%%%%%%%%%%%%%%%%%%%%%%
\subsection{编译} 
%%%%%%%%%%%%%%%%%%%%%%%%%%%%%%%%%%%%%%%%%%%%%%%%%%

\begin{LIST}{1cm}

  \IT{(\FU*{COMPILE} 
    \xorGOO{\NIL\VAR{ definition}\\
      \xorGOO{\VAR{name}\\
      (\kwd{setf } \VAR{name})}{\}}\text{ }\Op{\VAR{definition}}}{\}})}
  {
    返回 \retval{已编译函数} 或用已编译函数替换
    \retval{\VAR{name}} 的函数定义。若发生
    \kwd{warning}s 或 \kwd{error}s,返回
    \retvalii{\T}。若非 \kwd{style-warning}s 的
    \kwd{warning}s 或 \kwd{error}s,返回 \retvaliii{\T}。
  }

  \IT{(\FU*{COMPILE-FILE} \VAR{file} 
    \orGOO{\kwd{:output-file}\VAR{ out-path}\\
      \kwd{:verbose} \VAR{ bool}\DF{\V{\A compile-verbose\A}}\\
      \kwd{:print}\VAR{ bool}\DF{\V{\A compile-print\A}}\\
      \kwd{:external-format} \VAR{
        file-format}\DF{\kwd{:default}}}{\}})}
  {
    将 \VAR{file} 的编译结果写入 \VAR{out-path}。返回
    \retval{真实输出路径} 或 \retval{\NIL}。若发生 \kwd{warning}s 或
    \kwd{error}s,返回 \retvalii{\T}。若非 \kwd{style-warning}s 的
    \kwd{warning}s 或 \kwd{error}s,返回 \retvaliii{\T}。
  }

  \IT{(\FU*{COMPILE-FILE-PATHNAME} \VAR{file} \Op{\kwd{:output-file}
      \VAR{path}} \Op{\VAR{other-keyargs}})}
  {
  若以相同参数调用,\FU{COMPILE-FILE}
  将写入的 \retval{路径名}。
  }

  \IT{(\FU*{LOAD} \VAR{path} 
    \orGOO{\kwd{:verbose} \VAR{ bool}\DF{\V{\A load-verbose\A}}\\
      \kwd{:print} \VAR{ bool}\DF{\V{\A load-print\A}}\\
      \kwd{:if-does-not-exist} \VAR{ bool}\DF{\T}\\
      \kwd{:external-format} \VAR{
        file-format}\DF{\kwd{:default}}}{\}})}
  {
  加载源文件或编译文件入 Lisp 环境。若成功,返回
  \retval{\T}。
  }

  \IT{\arrGOO{\V{\A compile-file}\\
      \V{\A load}}{\}}\kwd{-}%
    \xorGOO{\kwd{pathname\A}\DF{\NIL}\\
      \kwd{truename\A}\DF{\NIL}}{.}}
  {\index{*COMPILE-FILE-PATHNAME*@\A COMPILE-FILE-PATHNAME\A}\index{*COMPILE-FILE-TRUENAME*@\A COMPILE-FILE-TRUENAME\A}\index{*LOAD-PATHNAME*@\A LOAD-PATHNAME\A}\index{*LOAD-TRUENAME*@\A LOAD-TRUENAME\A}
  \FU{compile-file}/\FU{load} 输入文件名。
  }

  \IT{\arrGOO{\V{\A compile}\\
      \V{\A load}}{\}}\kwd{-}%
    \xorGOO{\kwd{print\A}\\
      \kwd{verbose\A}}{.}}  
  {\index{*COMPILE-PRINT*@\A COMPILE-PRINT\A}\index{*COMPILE-VERBOSE*@\A COMPILE-VERBOSE\A}\index{*LOAD-PRINT*@\A LOAD-PRINT\A}\index{*LOAD-VERBOSE*@\A LOAD-VERBOSE\A}
  \FU{compile-file}/\FU{load} 默认值。
  }

  \IT{(\SO*{EVAL-WHEN} (%
    \orGOO{\Goo{\kwd{:compile-toplevel}\XOR\kwd{compile}}\\
      \Goo{\kwd{:load-toplevel}\XOR\kwd{load}}\\
      \Goo{\kwd{:execute}\XOR\kwd{eval}}}{\}}) \PROGN{\VAR{form}})} 
  {
  若 \SO{EVAL-WHEN}
  位于正在编译的文件顶层,正在加载的文件顶层,或任意位置,返回
  \retval{\VAR{form}s 的所有值}。若 \VAR{form}s
  不求值,返回 \retval{\NIL}。(\kwd{compile}、\kwd{load}
  及 \kwd{eval} 已废弃。)
  }

  \IT{(\SO*{LOCALLY} \OPn{(\kwd{declare} \OPn{\NEV{\VAR{decl}}})}
    \PROGN{\VAR{form}})}
  {
  在 \VAR{decl} 有效的词法环境中求值
  \VAR{form}s。返回 \retval{\VAR{form}s 的所有值}。
  }

  \IT{(\MC*{WITH-COMPILATION-UNIT} (\Op{\kwd{:override}
      \VAR{bool}\DF{\NIL}}) \PROGN{\VAR{form}})}
  {
  返回 \retval{\VAR{form}s 的所有值}。由编译器延迟至编译结束的警告延迟至
  \VAR{form}s
  求值结束。
  }

  \IT{(\SO*{LOAD-TIME-VALUE} \VAR{form}
    \Op{\NEV{\VAR{read-only}}\DF{\NIL}})}
  {
  编译时求值 \VAR{form},运行时将
  \retval{其值} 作为字面值。
  }

  \IT{(\SO*{QUOTE} \NEV{\VAR{foo}})}
  {
  返回 \retval{未经求值的 \VAR{foo}}。
  }

  \IT{(\GFU*{MAKE-LOAD-FORM} \VAR{foo} \Op{\VAR{environment}})}
  {
  Its methods are to return a \retval{creation form} which on
  evaluation at \FU{load} time returns an object equivalent to
  \VAR{foo}, and an optional \retvalii{initialization form} which on
  evaluation performs some initialization of the object. 
  }

  \IT{(\FU*{MAKE-LOAD-FORM-SAVING-SLOTS} \VAR{foo}
    \orGOO{\kwd{:slot-names} \VAR{ slots}\DF{all local slots}\\
      \kwd{:environment}  \VAR{ environment}}{\}})}
  {
  Return a \retval{creation form} and an \retvalii{initialization
    form} which on evaluation construct an object equivalent to
  \VAR{foo} with \VAR{slots} initialized with the corresponding values
  from \VAR{foo}. 
  }

  \IT{\arrGOO{(\FU*{MACRO-FUNCTION} \VAR{ symbol }
      \Op{\VAR{environment}})\\
      (\FU*{COMPILER-MACRO-FUNCTION}\text{ } 
      \xorGOO{\VAR{name}\\
        (\kwd{setf } \VAR{name})}{\}}\text{ }
      \Op{\VAR{environment}})}{.}}
  {
  Return specified \retval{macro function}, or \retval{compiler mac\-ro
    func\-tion}, respectively, if any. Return \retval{\NIL}
  otherwise. \kwd{setf}able. 
  }
  
  \IT{(\FU*{EVAL} \VAR{arg})}
  {
  返回在全局环境中求 \retval{\VAR{arg} 的值所得到的所有值}。
  }

\end{LIST}


%%%%%%%%%%%%%%%%%%%%%%%%%%%%%%%%%%%%%%%%%%%%%%%%%%
\subsection{REPL 及调试} 
%%%%%%%%%%%%%%%%%%%%%%%%%%%%%%%%%%%%%%%%%%%%%%%%%%

\begin{LIST}{1cm}
  
  \IT{\arrGOO{\V*{+}\XOR\V*{++}\XOR\V*{+++}\\[1pt]
      \V{\A}\text{ }\XOR\text{ }\V{\A\A}\text{ }\XOR\text{ }\V{\A\A\A}\\[1pt]
      \V*{/}\text{ }\XOR\text{ }\V*{//}\text{ }\XOR\text{ }\V*{///}}{.}}
  {\index{*@\A}\index{**@\A\A}\index{***@\A\A\A}
  最后、倒数第二、倒数第三个在 REPL
  中求值的 \retval{形式},或其 \retval{主值},或其所有值之
  \retval{列表}。
  }

  \IT{\V*{--}}
  {
  当前在 REPL 中求值的 \retval{形式}。
  }

  \IT{(\FU*{APROPOS} \VAR{string} \Op{\VAR{package}\DF{\NIL}})}
  {
  打印包含 \VAR{string} 的已持有符号。
  }

  \IT{(\FU*{APROPOS-LIST} \VAR{string} \Op{\VAR{package}\DF{\NIL}})}
  {
  包含 \VAR{string} 的 \retval{已持有符号列表}。
  }

  \IT{(\FU*{DRIBBLE} \Op{\VAR{path}})}
  {
  Save a record of interactive session to file at \VAR{path}. Without
  \VAR{path}, close that file.
  }

  \IT{(\FU*{ED} \Op{\VAR{file-or-function}\DF{\NIL}})}
  {
  如果可能,调用编辑器。
  }

  \IT{(\xorGOO{\FU*{MACROEXPAND-1}\\
      \FU*{MACROEXPAND}}{\}} \VAR{form} \Op{\VAR{environment}\DF{\NIL}})}
  {
  Return \retval{macro expansion}, once or entirely, respectively, of
  \VAR{form} and \retvalii{\T} if \VAR{form} was a macro form. 
  Return \retval{\VAR{form}} and \retvalii{\NIL} otherwise.
  }

  \IT{\V{\A macroexpand-hook\A}}
  {\index{*MACROEXPAND-HOOK*@\A MACROEXPAND-HOOK\A}
  Function of arguments expansion function, macro form, and
  environment called by \FU{macroexpand-1} to generate macro
  expansions. 
  }

  \IT{(\MC*{TRACE} \xorGOO{\VAR{function}\\
    (\kwd{setf } \VAR{function})}{\}^{\!\!*}})}
  {
  Cause \VAR{function}s to be traced. With no arguments,
  return \retval{list of traced functions}.
  }

  \IT{(\MC*{UNTRACE} \xorGOO{\VAR{function}\\
    (\kwd{setf } \VAR{function})}{\}^{\!\!*}})}
  {
  Stop \VAR{function}s, or each currently traced function, from being
  traced. 
  }

  \IT{\V{\A trace-output\A}}
  {\index{*TRACE-OUTPUT*@\A TRACE-OUTPUT\A}
    \MC{trace} 及 \MC{time} 所发送的输出流。
  }

  \IT{(\MC*{STEP} \VAR{form})}
  {
  Step through evaluation of \VAR{form}. Return \retval{values of
    \VAR{form}}. 
  }

  \IT{(\FU*{BREAK} \Op{\VAR{control} \OPn{\VAR{arg}}})}
  {
  Jump directly into debugger; return \retval{\NIL}.
  See page \pageref{section:Format}, \FU{format}, for \VAR{control}
  and \VAR{arg}s. 
  }

  \IT{(\MC*{TIME} \VAR{form})}
  {
  Evaluate \VAR{form}s and print timing information to
  \V{\A trace-output\A}. Return \retval{values of \VAR{form}}. 
  }
  \IT{(\FU*{INSPECT} \VAR{foo})}
  {
    交互式给出 \VAR{foo} 的信息。
  }

  \IT{(\FU*{DESCRIBE} \VAR{foo}
    \Op{\DES{\VAR{stream}}\DF{\V{\A standard-output\A}}})}
  {
  向 \VAR{stream} 发送 \VAR{foo} 的信息。
  }

  \IT{(\GFU*{DESCRIBE-OBJECT} \VAR{foo} \Op{\DES{\VAR{stream}}})}
  {
    向 \VAR{stream} 发送 \VAR{foo} 的信息。可由
    \FU{describe} 调用。
  }

  \IT{(\FU*{DISASSEMBLE} \VAR{function})}
  {
  发送 \VAR{function} 的反汇编结果至
  \V{\A standard-output\A}。返回 \retval{\NIL}。
  }

  \IT{(\FU*{ROOM} \Op{\Goo{\NIL\XOR\kwd{:default}\XOR\T}\DF{\kwd{:default}}})}
  {
    Print information about internal storage management to \kwd{\A
      standard-output\A}.
  }



\end{LIST}


%%%%%%%%%%%%%%%%%%%%%%%%%%%%%%%%%%%%%%%%%%%%%%%%%%
\subsection{声明}
%%%%%%%%%%%%%%%%%%%%%%%%%%%%%%%%%%%%%%%%%%%%%%%%%%
\begin{LIST}{1cm}

  \IT{\arrGOO{(\FU*{PROCLAIM} \VAR{ decl})\\
      (\MC*{DECLAIM } \OPn{\NEV{\VAR{decl}}})}{.}}
  {
  全局声明
  \VAR{decl}。\VAR{decl}
  可以是:\kwd{declaration}、\kwd{type}、\kwd{ftype}、\kwd{inline}、\kwd{notinline}、\kwd{optimize}、\kwd{special}。见下。
  }

  \IT{(\kwd*{DECLARE} \OPn{\NEV{\VAR{decl}}})}
  {
    在某些形式内,局部声明
    \OPn{\VAR{decl}}。\VAR{decl}
    可以是:\kwd{dynamic-extent}、\kwd{type}、\kwd{ftype}、\kwd{ignorable}、\kwd{ignore}、\kwd{inline}、\kwd{notinline}、\kwd{optimize}、\kwd{special}。见下。
  }

  \begin{LIST}{.5cm}
    
    \IT{(\kwd*{DECLARATION} \OPn{foo})}
    {
      Make \VAR{foo}s names of declarations.
    }
 
    \IT{(\kwd*{DYNAMIC-EXTENT} \OPn{\VAR{variable}} \OPn{(\kwd{function}
        \VAR{function})})}
    {
      Declare lifetime of \VAR{variable}s and/or \VAR{function}s to end
      when control leaves enclosing block.
    }


    \IT{\arrGOO{(\Op{\kwd*{TYPE}} \VAR{ type} \OPn{\VAR{ variable}})\\
        (\kwd*{FTYPE} \VAR{ type} \OPn{\VAR{ function}})}{.}}
    {
      声明 \VAR{variable}s 或 \VAR{function}s 为 \VAR{type}。
    }

    \IT{(\xorGOO{\kwd*{IGNORABLE}\\
        \kwd*{IGNORE}}{\}}
      \xorGOO{%
        \VAR{var}\\
        (\kwd{function} \VAR{ function})}{\}^{\!\!*}})}
    {
    关闭已使用/未使用绑定的警告。
  }

    \IT{\arrGOO{(\kwd*{INLINE} \OPn{\VAR{ function}})\\
        (\kwd*{NOTINLINE} \OPn{\VAR{ function}})}{.}}
    {
    Tell compiler to integrate/not to integrate, respectively, called
    \VAR{function}s into the calling routine.
  }

    \IT{(\kwd*{OPTIMIZE} \orGOO{%
        \kwd*{COMPILATION-SPEED}\XOR(\kwd*{COMPILATION-SPEED}\VAR{ n}\DF{\LIT{3}})\\
        \kwd*{DEBUG}\XOR(\kwd*{DEBUG}\VAR{ n}\DF{\LIT{3}})\\
        \kwd*{SAFETY}\XOR(\kwd*{SAFETY}\VAR{ n}\DF{\LIT{3}})\\
        \kwd*{SPACE}\XOR(\kwd*{SPACE}\VAR{ n}\DF{\LIT{3}})\\
        \kwd*{SPEED}\XOR(\kwd*{SPEED}\VAR{ n}\DF{\LIT{3}})}{\}})}
    {
    指定编译器优化方式。$n=0$ 代表不重要,$n=1$
    代表一般,$n=3$ 代表重要。
  }

    \IT{(\kwd*{SPECIAL} \OPn{\VAR{var}})}
    {
    声明 \VAR{var}s 为动态变量。
  }

  \end{LIST}
\end{LIST}




% LocalWords:  mac ro func tion

%%% Local Variables: 
%%% mode: latex
%%% TeX-master: "clqr"
%%% End: 
